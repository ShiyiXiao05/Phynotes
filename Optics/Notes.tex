\documentclass[12pt,a4paper,oneside]{report}

% ========== 中文与页面设置 ==========
\usepackage[UTF8]{ctex}
\usepackage{geometry}
\geometry{a4paper, margin=2.5cm}
\usepackage{bm}
\usepackage{mathrsfs}
\usepackage{bm}
\usepackage{multirow}
\usepackage{array}
% ========== 数学与定理环境 ==========
\usepackage{amsmath,amssymb,amsthm,bm}
\usepackage{physics} % 物理公式快捷命令
\newtheorem{theorem}{定理}[chapter]
\newtheorem{lemma}{引理}[chapter]
\newtheorem{corollary}{推论}[chapter]
\theoremstyle{definition}
\newtheorem{definition}{定义}[chapter]
\newtheorem{example}{例题}[chapter]
\theoremstyle{remark}
\newtheorem*{remark}{注}

\setlength{\headheight}{15pt} % 将页眉高度设置为15pt
\addtolength{\topmargin}{-3pt} % 同时向上收缩页边距以补偿布局


\usepackage{fontspec}
\usepackage{xeCJK}

% 文楷 GB 字体族(不作为主字体)
\setCJKfamilyfont{wenkai}[
    Path=fonts/,
    Extension = .ttf,
    UprightFont = *-Regular  % 指向 LXGWWenKaiGB-Regular.ttf
]{LXGWWenKaiGB}
\newcommand{\wenkai}{\CJKfamily{wenkai}}
% ========== 图表/代码/颜色支持 ==========
\usepackage{graphicx}
\usepackage{caption}
\newcommand{\insertfig}[3]{
    \begin{figure}[ht]
        \centering
        \includegraphics[width=#3\textwidth]{#1}
        \caption{#2}
        \label{fig:#1}
    \end{figure}
}
\newcommand{\insertpic}[3]{
    \begin{center}
        \includegraphics[width=#3\textwidth]{#1}
        
        \par % <--- 关键修改1:强制图片后结束段落,确保标题在下一行
        \vspace{4pt} 
        
        \refstepcounter{figure} 
        
        % 关键修改2:使用 \mbox{...} 包裹整个标题
        % \mbox 的作用是禁止盒子内的文字换行
        {\wenkai\small \mbox{图 \thefigure\quad #2}} 
        
        \label{fig:#1}
    \end{center}
}
\usepackage{float}
\usepackage{caption}
\usepackage{subcaption}
\usepackage{booktabs}
\usepackage{xcolor}
\usepackage{listings}
\usepackage{esint}
\graphicspath{{figure/}}
\newcommand{\mb}[1]{\bm{#1}}
\lstset{
  language=Python,
  basicstyle=\ttfamily\small,
  keywordstyle=\color{blue},
  commentstyle=\color{gray},
  stringstyle=\color{orange},
  frame=single,
  breaklines=true
}


% -------------------------------
% 注记环境(小字、楷体、灰框)
% -------------------------------
\usepackage{tcolorbox}
\tcbuselibrary{skins,breakable}

\newenvironment{noteenv}{
  \begin{tcolorbox}[
      colback = gray!5,
      colframe = gray!40,
      breakable,
      enhanced,
      sharp corners,
      left=6pt,right=6pt,top=6pt,bottom=6pt,
  ]
  \small
  \CJKfamily{wenkai} % 👈 只这里使用文楷GB
}{
  \end{tcolorbox}
}

% ========== 目录与超链接 ==========
\usepackage{hyperref}
\hypersetup{
  colorlinks=true,
  linkcolor=blue,
  urlcolor=blue,
  citecolor=magenta
}
\usepackage{tocloft}

% ========== 页眉页脚 ==========
\usepackage{fancyhdr}
\pagestyle{fancy}
\fancyhf{}
\fancyhead[L]{光学笔记}
\fancyhead[R]{\leftmark}
\fancyfoot[C]{\thepage}

% ========== 章节格式美化 ==========
\usepackage{titlesec}
\titleformat{\chapter}[hang]{\Huge\bfseries}{第\thechapter 章}{1em}{}
\titleformat{\section}[hang]{\Large\bfseries}{\thesection}{0.5em}{}
\titleformat{\subsection}[hang]{\large\bfseries}{\thesubsection}{0.5em}{}

% ========== 参考文献 ==========
\usepackage[numbers]{natbib}

% ========== 封面信息 ==========
\title{\Huge\textbf{Lecture Notes on Optics}}
\author{\Large 上海交通大学 人工智能学院 \\
\Large 肖世屹}
\date{\today}


\everymath{\displaystyle}
\renewcommand{\d}{\mathop{}\!\mathrm{d}}
\renewcommand{\v}{\mathop{}\!\varepsilon}
% =======================================================
\begin{document}

% --------------------- 封面 ---------------------
\maketitle
\thispagestyle{empty}
\clearpage
% --------------------- 序言 ---------------------
\chapter*{序言}
\addcontentsline{toc}{chapter}{序言} % 让“序言”出现在目录中

在本笔记中,我们系统地整理了光学的主要内容,包括波动光学引论、光的干涉、光的衍射、光的偏振等四个章节。本笔记的内容基于上海交通大学大学物理(荣誉)II的课程考试要求做了适当的调整,使其学习难度和内容适于准备考试。本笔记现已发布在 Github 上,读者可以访问\url{https://github.com/ShiyiXiao05/Phynotes}以获取更新。

在本笔记中,用注记环境排版了一些有别于正文的内容,这些内容超出了课程考试要求的范围,但它们对于理解课程内容有所帮助,读者阅读可以按照需要有选择地学习。

在编写笔记的过程中,感谢我的室友陈志杰给予的排版支持,在他的帮助下这本笔记得以以美观的形式呈现给大家。同时也要感谢韩岳成、何忠颐等同学的试读和勘误。

希望本笔记能为学习光学的读者提供一定帮助。若能抛砖引玉,为读者揭开物理之美的一角,作者将不胜欣慰。

本笔记系基于陈洁和胡其图老师讲授的《大学物理(荣誉)》课程内容整理而成,文中部分图表、图片及例题素材直接引用授课课件,其原始内容的知识产权归原作者所有。本文档完全开源免费,仅供个人学习使用,严禁用于任何形式的商业盈利活动。如权利人认为本笔记中的引用不当或侵犯了您的权益,请通过 sjtuxsy-finance-pc@sjtu.edu.cn 联系本人,本人将在核实后第一时间删除相关内容。

由于时间与作者水平所限,文中难免存在疏漏与错误,恳请各位读者批评指正。本人不胜感谢。

\bigskip
\begin{flushright}
肖世屹 \\
上海交通大学 人工智能学院 \\
\today
\end{flushright}
\clearpage

% --------------------- 目录 ---------------------
\tableofcontents
\clearpage

% --------------------- 正文章节 ---------------------

\graphicspath{{figure/chap1/}}
\chapter{波动光学基本原理}
\section{惠更斯原理}
惠更斯在距今三百多年前提出了一个关于光波传播的概念,其大意如下:
光扰动同时到达的空间曲面被称为波阵面或者波前, 波前上的每一点都可以被看作一个新的扰动中心, 称其为子波源或次波源,次波源向四周激发次波; 下一时刻的波前应当是这些大量次波面的公共切面,也称其为包络面;次波中心与其次波面上的那个切点的连线方向,给出了该处光传播方向,亦即光射线方向.

根据惠更斯原理,人们可以由$t$时刻的波前,用作图法导出下一时刻$t+\Delta t$的波前.这个原理解决了波前随时间在空间中传播的问题。

\begin{figure}[htbp]
    \centering
    % 第一张图
    \begin{minipage}[b]{0.48\textwidth}
        \centering
        \includegraphics[width=0.55\textwidth]{1-1.png} % figure/image1.png
        \caption{均匀介质中光沿直线传播}
    \end{minipage}
    \hfill
    % 第二张图
    \begin{minipage}[b]{0.48\textwidth}
        \centering
        \includegraphics[width=0.55\textwidth]{1-2.png} % figure/image2.png
        \caption{非均匀介质中光线弯曲}
    \end{minipage}
\end{figure}

惠更斯原理可以得到一系列有意义的结果,这里以折射定律为例说明.如图,设两种介质的界面为平面,上方光速为$v_1$,下方光速为$v_2$,入射角为$i_1$.取与入射光束正交的平面$ABC$作为波前,则该波前上经过$C$点的光线到达入射点$C'$的时间为$\frac{CC'}{v_1}$.以波前上的$A$点为次波源,产生的次波的波面半径为$\rho_A=v_2 \Delta t$.在$\Delta t$时间内,波前上经过点$B$的光线传播到入射点$B'$后会有相应的按比例缩小的次波球面。这样对次波球面作切线$C'A'$,不难证明$C'A'$也相切于按比例缩小的一系列次波面上,这就得到了一个存在于介质2中的新波前$C'B'A'$.作新波前的法线,这方向即为折射光线方向. 由几何关系可得:
\[
\sin i_1 = \frac{CC'}{AC} = \frac{v_1 \Delta t}{AC}, \quad \sin i_2 = \frac{AA'}{AC} = \frac{v_2 \Delta t}{AC}
\] 
注意到$CC'=v_1\Delta t$,$AA'=v_2\Delta t$,于是:
\[
\frac{\sin i_1}{\sin i_2} = \frac{v_1}{v_2}
\]

\insertfig{1-3.png}{由惠更斯原理导出折射定律}{0.35}

这样基于折射定律,我们可以定义折射率$n$:
\[
\frac{\sin i_1}{\sin i_2} = n_{12} = \frac{n_2}{n_1}
\]

如果我们定义真空中的折射率为$1$,则介质中的折射率为:
\[
n = \frac{c}{v}
\]

扰动在空间中传播而形成波动,波速$v$等于扰动的时间频率$f$与波动的空间周期即波长$\lambda$的乘积,$v=f\lambda$.设真空中的光速为$c=f_0\lambda_0$, 则折射率可以被表示为:
\[
n = \frac{f_0 \lambda_0}{f \lambda}
\]
问题是,介质中光的频率$f$是否等于真空中的频率$f_0$?在线性介质的光场中,光扰动的时间频率仅由光源决定,与介质无关。这样我们有$f=f_0$, 折射率可以被简化为:
\[
n = \frac{\lambda_0}{\lambda}
\]
这表明在介质中光的波长变短了。介质中的光的波长虽然变化,其频率保持不变,这也解释了为什么我们没有观察到介质中光的颜色变化,因为真正与视网膜上视觉细胞相互作用的是光的扰动,而决定扰动的时间频率是保持不变由光源决定的。

\section{费马原理}
我们定义光纤路径的几何长度与经过的介质折射率的乘积被定义为光程:
\[
L = \int_A^B n \d s
\]

光程的物理意义可以由下面的这个式子来说明。考察一列由$A$传播到$B$的光波,我们知道沿着光传播的方向,各点扰动的相位是逐点落后的,在单位长度上的相位落后由波数$k=\frac{2\pi}{\lambda}$决定,即:
\[
\varphi(B)-\varphi(A) = -\int_A^{B}\frac{2\pi}{\lambda} \d s 
\]
我们利用折射率的定义$n=\frac{\lambda_0}{\lambda}$,可以将上式改写为:
\[
\varphi(B)-\varphi(A) = -\frac{2\pi }{\lambda_0}\int_A^B  n\d s = -\frac{2\pi}{\lambda_0} L
\]
这表明光程代表了光扰动在传播时的相位变化,光程越大,相位落后越多.

费马原理是说,在光场中从$A$到$B$有一条实际光线$l_0$,这个几何路径使得光程$L$取得平稳值:
\[
\int_{l_0} n \d s = \text{extreme}
\]

路径积分为平稳值,有三种情况,分别是取极小值——这最为常见; 取常数——成像时物像关系; 取极大值——这较为少见.

由费马原理可以导出直线传播定律,反射定律和折射定律。欧氏空间中两点间直线距离最短,折射率为常数时显然应该沿此路径传播。

\begin{figure}[htbp]
    \centering
    % 第一张图
    \begin{minipage}[b]{0.48\textwidth}
        \centering
        \includegraphics[width=0.55\textwidth]{1-4.png} % figure/image1.png
        \caption{费马原理导出反射定律}
    \end{minipage}
    \hfill
    % 第二张图
    \begin{minipage}[b]{0.48\textwidth}
        \centering
        \includegraphics[width=0.55\textwidth]{1-5.png} % figure/image2.png
        \caption{费马原理导出折射定律}
    \end{minipage}
\end{figure}

对于反射情形,引入光源关于反射面的镜面对称点$Q'$,这样光程为极小值的条件显然是$Q'MP$为一条直线而非折线,因此我们得到入射角等于反射角。

对于折射情形,设定光源$Q$和观察点$P$,界面上的动点$M$为待定的入射点,考察光程:
\[
L(QMP) = n_1 QM + n_2 MP = n_1 \sqrt{a^2+x^2} + n_2 \sqrt{b^2+(d-x)^2}
\]

在这一元情况下取极值要求
\[
\frac{\d L}{\d x} = 0 \Rightarrow \frac{n_1 x}{\sqrt{a^2+x^2}} = \frac{n_2 (d-x)}{\sqrt{b^2+(d-x)^2}} \Rightarrow n_1 \sin i_1 = n_2 \sin i_2
\]

上式中已经利用
\[
    \sin i_1 = \frac{x}{\sqrt{a^2+x^2}}, \quad \sin i_2 = \frac{d-x}{\sqrt{b^2+(d-x)^2}}
\]

\begin{noteenv}
Richard Feynman 有更加独特的方式来看待镜面反射现象。Feynman 通过路径积分的方式利用许多小箭头解释了反射现象。Feynman提出,光在参与反射时会经过所有可能的路径,但镜子上的每一个点对于最终振幅的贡献是不一样的。如图(a)所示,我们将镜面分为许多方块,每一个方块都对应着一条真实的光路。显然这些光路对应的传播时间是不一样的。现在让我们设想每个光子从光源出发时都携带了一块秒表,其指针随着光的传播而转动,初始时刻所有的秒表指针都指向同一方向,在到达终点时停止计时,这样箭头的方向就携带了相位信息。我们用一系列箭头来代表各个方块对于最终的振幅贡献,每个方块上的振幅大小相同,因此各箭头的长度相同,其方向如图(c)所示。将他们叠加起来,最终得到的结果如图(d)所示。

\begin{center}
\begin{minipage}[t]{0.48\linewidth}
    \centering
    \includegraphics[width=\linewidth]{1-6.png}
\end{minipage}
\hfill
\begin{minipage}[t]{0.48\linewidth}
    \centering
    \includegraphics[width=\linewidth]{1-7.png}
\end{minipage}

\end{center}
我们发现对最终的振幅贡献最大的箭头位于中心,这是因为此处光程取得极值,相邻箭头间的方向变化不大,相位差距不大,叠加后仍有较大贡献。而位于两端的相邻箭头间的相位差距较大,导致它们的贡献相互抵消。这样Feynman说,费马原理指出的光程取稳定值是因为这条路径的及其邻近路径的光对于最终的振幅贡献最大。这样的小箭头方法在我们处理衍射时也有一定意义。
\end{noteenv}

\section{光的叠加原理}
光是一种特定形式的电磁波,可见光的波长$\lambda$约在$380\mathrm{nm}$到$760\mathrm{nm}$之间,其对应的频率大约是$4 \times 10^{14}$到$8 \times 10^{14}$ Hz。Maxwell方程告诉我们,各种形式的电磁波的传播速度都被波速方程制约为:
\[
v = \frac{1}{\sqrt{\mu \epsilon}} \Rightarrow n = \frac{c}{v} =  \frac{\sqrt{\mu \epsilon}}{\sqrt{\mu_0 \varepsilon_0}} = \sqrt{\varepsilon_r \mu_r}
\]

关于折射率的微观机制和性质的研究都是从上式出发的。

平面电磁波是自由空间电磁波的一个基元成分。我们一般用电磁波的电场矢量来代表电磁波。这样平面电磁波可以被表示为:
\[
\bm{E}(\bm{r}, t) = \bm{E}_0 \cos (\omega t - \bm{k} \cdot \bm{r}+\varphi_E)
\]

光是一种电磁波,而电磁波是一种横波。在与传播方向正交的平面上,$\bm{E}$和$\bm{H}$矢量的振动具有两个自由度,这种平面上的方向自由度被称为光的偏振结构,这也是偏振光的基础。

广义上说,扰动在空间中的传播也就是运动状态在空间中的传播形成波动。扰动同时到达的空间中各点形成一个等相位面。按照等相位面的形状,产生了各种对波的称谓,例如平面波,球面波等等。

按照时间尺度衡量,波可以被分为定态波和脉冲波。若在观测时间内光源持续且稳定地发光,则波场中各点均以同一频率作稳定的振荡,这种波被称为定态波。而光源在极短时间中发光,以至于波形局限于一个小区域内时,这种波被称为脉冲波。对于可见光,其周期$T\approx 10^{-14}\mathrm{s}$,普通光源从微观时间尺度看其一次持续发光时间量级$\tau\approx10^{-8} \mathrm{s}$。一个长波列中大约含有$10^6$个周期,这种情况就可以视为定态了。发光的机制我们在讨论光的时间相干性时还会遇见。

现在我们考虑光波的相关叠加。设在同一介质中有两列平面电磁波传播,分别为:
\[
E_1 = \bm{E_{10}} \cos (\omega t - \bm{k}_1 \cdot \bm{r} + \varphi_1), \quad
E_2 = \bm{E_{20}} \cos (\omega t - \bm{k}_2 \cdot \bm{r} + \varphi_2)
\]

叠加后合成的光矢量为:
\[
E = E_1 + E_2 = \bm{E_{10}} \cos (\omega t - \bm{k}_1 \cdot \bm{r} + \varphi_1) + \bm{E_{20}} \cos (\omega t - \bm{k}_2 \cdot \bm{r} + \varphi_2)
\]

注意到能流密度矢量对时间取平均得到波的强度,
\[
I = \langle S \rangle = \frac{1}{2} c \varepsilon_0 n E_0^2 \propto E_0^2
\]

这样将上式两端对观察时间求平均得到:
\[
I = \langle \bm{E}^2 \rangle = \langle \bm{E_1}^2 \rangle + \langle \bm{E_2}^2 \rangle + 2\langle \bm{E_1} \cdot \bm{E_2} \rangle = I_1 + I_2 + 2\langle \bm{E_1} \cdot \bm{E_2} \rangle
\]
$2\langle\bm{E_1} \cdot \bm{E_2}\rangle$被称为干涉项。

这里我们先不考虑偏振,认为
\[
\bm{E_1} \parallel \bm{E_2}
\]
从而我们可以写出干涉项的表达式为
\[
\begin{aligned}
2\langle \bm{E_1} \cdot \bm{E_2} \rangle & = 2 E_{10} E_{20} \langle \cos(\omega_1 t-\bm{k_1}\cdot \bm{r_1}+\varphi_{10})\cos(\omega_2 t -\bm{k_2}\cdot \bm{r_2}+\varphi_{20}) \rangle \\
& = E_{10}E_{20}\langle \cos((\omega_1+\omega_2) t -  (\varphi_{1} + \varphi_{2})) \rangle+ \langle \cos((\omega_1-\omega_2) t - (\varphi_{1} - \varphi_{2})) \rangle
\end{aligned}
\]
我们把光的空间传播带来的相位落后归结到$\varphi_1, \varphi_2$中。上式中前一项取时间平均显然为零,后一项在$\omega_1 = \omega_2$时不为零:
\[
2\langle \bm{E_1} \cdot \bm{E_2} \rangle = 2 E_{10} E_{20} \langle \cos(\Delta \varphi(P)) \rangle
\]
其中相位差$\Delta \varphi(P)$为
\[
\Delta \varphi(P) = \varphi_1 - \varphi_2 = \varphi_{10} - \varphi_{20}-k(r_1-r_2)
\]
这里把相位差写成场点$P$的函数意在强调正是相位差的空间变化决定了干涉条纹的形状和分布。

注意到只有在$\varphi_{10}-\varphi_{20}$恒定时,对相位差的时间平均才不为零,于是我们可以总结发生干涉的四个条件:
\begin{itemize}
    \item 光源的频率相同,即$\omega_1 = \omega_2$。
    \item 光源的相位差$\varphi_{10} - \varphi_{20}$恒定。
    \item 两列光波的振动方向相同或有固定关系。
    \item 两列光波存在时空重叠。
\end{itemize}

对于取时间平均意义来说,光源的相位差的绝对值并不重要,不妨取$\varphi_{10}=\varphi_{20}$。这样相位差可以被表示为:
\[
\Delta \varphi(P) = k(r_1 - r_2) = \frac{2\pi}{\lambda}(r_1 - r_2) = \frac{2\pi}{\lambda} \delta
\]
其中$\delta$为两列光波的光程差。注意到:
\[
\Delta \varphi=
\begin{cases}
\pm 2k\pi, & \text{当 } \delta = k\lambda, k \in \mathbb{Z}, \cos(\Delta \varphi)= 1, \text{干涉光强极大} \\ 
\pm (2k-1)\pi, & \text{当 } \delta = (k-\frac{1}{2})\lambda,k\neq 0, k \in \mathbb{Z}, \cos(\Delta \varphi)= -1, \text{干涉光强极小} \\
\end{cases}
\]
考虑介质的折射率不为1时,根据我们前面对光程的讨论,我们可以得到:
\[
\Delta \varphi = \frac{2\pi}{\lambda} n (r_1 - r_2) = \frac{2\pi}{\lambda_n} (r_1 - r_2) = \frac{2\pi}{\lambda_n} \delta
\]
其中$\lambda_n = \frac{\lambda}{n}$是介质中的波长。于是用波程差表达的干涉条件可以被写为:
\[
\begin{cases}
\delta = k \lambda_n, & k \in \mathbb{Z}, \quad \text{干涉极大} \\
\delta = (k - \frac{1}{2}) \lambda_n, & k \in \mathbb{Z}, \quad \text{干涉极小}
\end{cases}
\]


总结一下,我们可以得到用光强表达的双光束干涉强度公式:
\[
I = I_1 + I_2 + 2\sqrt{I_1 I_2} \cos(\Delta \varphi(P))
\]

用振幅表达的双光束干涉强度公式:
\[
I = A_1^2 + A_2^2 + 2A_1 A_2 \cos(\Delta \varphi(P))
\]

用相位差表示的干涉条件:
\[
\begin{cases}
\Delta \varphi = 2k\pi, & k \in \mathbb{Z}, \quad \text{干涉极大} \\
\Delta \varphi = (2k-1)\pi, & k \neq 0, k \in \mathbb{Z}, \quad \text{干涉极小}
\end{cases}
\]

用波程差表示的干涉条件为:
\[
\begin{cases}
\delta = k \lambda_n, & k \in \mathbb{Z}, \quad \text{干涉极大} \\
\delta = (k - \frac{1}{2}) \lambda_n, & k \neq 0,k \in \mathbb{Z}, \quad \text{干涉极小}
\end{cases}
\]

\graphicspath{{figure/chap2/}}
\chapter{光的干涉}
\section{杨氏双缝干涉}
用两个点波源作光的干涉实验的典型代表是杨氏双缝实验。

\insertfig{2-1.png}{杨氏双缝实验示意图}{0.7}

在普通单色光源前面放一个开有小孔$S$的屏,作为单色点光源。在$S$的照明范围内再开放一个开有小孔$S_1$和$S_2$的屏。按照惠更斯原理,$S_1$和$S_2$作为两个次波源向前发射次波(球面波),形成交叠的波场。在较远的地方放置一接收屏,屏上可以观测到一组几乎是平行的直线条纹。设从$S$到$S_1$,$S_2$的距离为$R_1$,$R_2$,光源$S$发光时的随机相位为$\varphi_0(t)$,$S_1$和$S_2$两个次波源的相位差为:
\[
\Delta \varphi_0 = \frac{2\pi}{\lambda}(R_2-R_1) +\varphi_0(t)-\varphi_0(t) = \frac{2\pi}{\lambda}(R_2-R_1)
\]

注意到随机相位在相减中被消去,保证了两束光分开后相位差的稳定性。像这样,将一束光通过装置分为两条传播路径,从而实现空间中的交叠和干涉,这种干涉被称为分波前(波阵面)的干涉。

\insertfig{2-2.png}{杨氏双缝实验结果}{0.5}

如图所示,光程差可以被表示为:
\[
\delta = r_2 - r_1 \approx d \sin \theta, \quad x =L\tan\theta \approx L \sin \theta \Rightarrow \delta = d \frac{x}{L}
\]

对于光强,$I_1 = I_2 = I_0 $,$\Delta \varphi(P) = \frac{2\pi}{\lambda} d\sin\theta$这样我们得到之前的光强表示公式为:
\[
I = I_1 + I_2 + 2\sqrt{I_1 I_2} \cos(\Delta \varphi(P)) = 2I_0+2I_0 \cos\left(\frac{2\pi}{\lambda} d \sin \theta \right) = 4I_0\cos^2\left(\frac{\pi}{\lambda} d \sin \theta \right)
\]

根据用光程差表示的干涉条件,我们可以得到:
\[
\begin{cases}
d \frac{x}{L} = k \lambda, & k \in \mathbb{Z}, \quad \text{干涉极大} \\
d \frac{x}{L} = (k - \frac{1}{2}) \lambda, & k \in \mathbb{Z}, \quad \text{干涉极小}
\end{cases}
\]

于是得到明纹和暗纹中心的位置为
\[
\begin{cases}
x = \pm k\frac{L}{d}\lambda ,& k \in \mathbb{N}, \text{明纹中心}\\
x = \pm(k - \frac{1}{2})\frac{L}{d}\lambda, & k \in \mathbb{N*},\text{暗纹中心}
\end{cases}
\]

相邻的明(暗)条纹间距为:
\[
\Delta x = \frac{L}{d}\lambda
\]

以上我们讨论的是单色光的干涉。如果光源中包含着两种波长的谱线,那么屏幕上将会存在两套间距不等的干涉条纹。

在讨论干涉装置的时候,我们往往会关心如下一些特征:
\begin{itemize}
    \item 相位差的计算公式;
    \item 条纹的特点:包括级次分布,中心条纹位置,条纹间距等等;
    \item 条纹的决定因素,变化特点等。
\end{itemize}

因此,现在我们来杨氏双缝实验中干涉条纹的移动。造成条纹移动的因素一般有三种,一是光源的移动,二是装置结构的变化,三是光路中介质的变化。

探讨干涉条纹的变动时通常可以用两种方式提出问题,一种是固定场点$P$,考察有多少条干涉条纹移过该点,另一种是跟踪干涉场中某一级条纹,例如0级条纹,看其朝什么方向移动了多少距离。对于前一种情况,我们需要靠擦和交于该点的两条相干光纤之间的光程差的表达式,每当光程差变化一个波长时,$P$点就会有一条干涉条纹移过。对于后一种情况,我们需要计算出该条纹的光程差,然后考察具有该光程差的场点的去向。

\begin{noteenv}
    我们来看看轴上点源移动对干涉条纹的影响。初始状态下零级条纹位于$O$点,光程差为零。现在将点源$Q$向$x$轴正方向移动了$x_0$,其造成的干涉条纹,其形状和条纹间距均不改变,但条纹中心发生了移动。零级条纹对应光程差为$0$.因此向上移动后,中心条纹处要求两条光线的光程相等,即为:
    \[
    r_1 + R_1 = r_2 + R_2
    \]
    注意到两孔左右的几何结构的相似性,可以得到:
    \[
    R_2 - R_1 = \frac{d}{R}x_0 , \quad r_2 - r_1 = \frac{d}{D}\delta x
    \]
    于是得到移动的距离$\delta x$和移动的条纹数量是:
    \[
    \delta x = \frac{D}{R}x_0, \quad N = \frac{\delta x}{\Delta x} = \frac{d}{R\lambda}x_0
    \]
    \insertpic{2-5.png}{点源移动对于干涉条纹的影响}{0.4}
\end{noteenv}

下面再来看些分波阵面干涉装置。分波阵面干涉的装置都可以被最终等效为杨氏双缝干涉装置,我们所需要做的就是计算清楚各物理量的大小。首先来看菲涅耳双面镜。点光源$Q$在双面镜的反射下会成$S_1$,$S_2$两个虚像,这三点位于同一个$M$为圆心的圆上。设双面镜之间的夹角为$\alpha$,点源与$M$之间的距离为$B$,交棱与屏幕的距离为$C$,那么对应于杨氏双缝实验,得到:
\[
d = 2\alpha B, \quad L =B+C \Rightarrow \Delta x = \frac{(B+C)\lambda}{2\alpha B}
\]

\insertfig{2-3.png}{菲涅耳双面镜示意图}{0.4}

其次来看劳埃德镜。点光源$S$在平面镜的反射下会成$S'$的虚像,$S$和$S'$之间的距离为$2h$,屏幕距离镜面为$D$,那么对应于杨氏双缝实验,得到:
\[
\Delta x = \frac{D\lambda}{2h}
\]

\insertfig{2-4.png}{劳埃德镜示意图}{0.4}

\section{薄膜干涉}

薄膜干涉是指如图所示的装置。从点光源发出的光照射到一层厚度较小的膜上,在上表面发生反射的光与在薄膜内发生一次反射后透射出来的光相互干涉的现象。

\insertfig{2-6.png}{薄膜干涉示意图}{0.4}

从点光源$Q$发出的两条特定光线交于点$P$,在$P$点后我们常常使用各种观察装置对条纹进行观察。从$Q$到$P$的这段距离内,光程差为:
\[
\Delta L = n_1 QA +n AB +n BP - n_1 QP
\]

注意到膜很薄,$\Delta \theta$很小,作一级近似,有$QA\approx QC$,又注意到:$AB+AP=\frac{2h}{\cos i}$,折射定律:$n_1\sin i_1= n\sin i$.
则:
\[
\Delta L = \frac{2nh}{\cos i}-n_1AP\sin i_1 = \frac{2nh}{\cos i}-nAP\sin i = \frac{2nh}{\cos i}-n(2h\tan i)\sin i = 2nh\cos i
\]

这样我们就计算出了薄膜干涉的光程差表示式。

光在介质界面发生反射时可能存在相位的突变,这种突变带来的光程差等效为半个波长,因此往往被称为半波损失。对半波损失的原理我们不做讨论,仅仅给出结论如下:在光从光疏介质向光密介质传播发生反射时有半波损失,反之无半波损失,任何情况下透射光均无半波损失。

在干涉问题中我们往往关心从上下表面反射的两光束之间是否因为半波损失而出现额外的半波光程差,对此问题而言,结论是当$n_1<n>n_2$或$n_1>n<n_2$时,存在半波损失,反之$n_1>n>n_2$或者$n_1<n<n_2$时无半波损失。

\subsection{等厚干涉}

薄膜表面干涉条纹的形状与观察和照明的方式有很大关系,我们讨论正入射方式,也就是入射光和反射光处处与表面垂直。这时$i=0$,因此
\[
\Delta L = 2nh
\]
也就是说下表面反射的光比上表面反射的光多走的路程就是前者在薄膜内部一次垂直的往返。由上式我们看到光程差只与厚度有关,因此光强的分布也取决于厚度$h$,薄膜上厚度相等各点的轨迹被称为等厚线。由于相邻条纹上的光程差为一个波长,因此相邻等厚条纹对应的厚度差为
\[
\Delta h = \frac{\lambda}{2n}
\]

我们现在来介绍劈尖干涉。其由两个不平行的反射平面组成,实际中往往是两个平板玻璃。这种装置的等厚线是一组平行于棱的直线。

\insertfig{2-7.png}{劈尖干涉示意图}{0.4}

设劈尖的劈角为$\alpha$,注意到相邻条纹间的厚度为$\Delta h = \frac{\lambda}{2}$,且几何关系为:$\Delta x =\frac{\Delta h}{\alpha}$,代入得到楔形膜表面条纹的间距为:
\[
\Delta x =\frac{\lambda}{2\alpha}
\]
在我们讨论的情况下,该种情况下会发生半波损失。因此劈尖光程差为0处是暗纹。

我们可以利用等厚干涉来进行多种测量。

\begin{figure}[htbp]
    \centering
    % 第一张图
    \begin{minipage}[b]{0.45\textwidth}
        \centering
        \includegraphics[width=0.55\textwidth]{2-8.png} % figure/image1.png
        \caption{测定滚珠直径}
    \end{minipage}
    \hfill
    % 第二张图
    \begin{minipage}[b]{0.45\textwidth}
        \centering
        \includegraphics[width=0.55\textwidth]{2-9.png} % figure/image2.png
        \caption{测量细丝直径}
    \end{minipage}
\end{figure}

劈尖厚度的微小变化或者劈角的微小变化,均将引起表面条纹发生显著的变化。设想我们整体将上平板向上平移,此时劈角不变,因此条纹间距不会发生变化,而是干涉条纹整体发生平移。为了考察平移的方向,我们选择考察具有固定光程差的条纹的位置。如图所示,假定交棱在左边,则此种情况下,具有相同光程差的位置将想左移动,同样的分析可以得到,条纹整体平移总是向着楔形交棱那一方。当膜厚度增加时,其定量关系是,若条纹移动过$N$条,则厚度变化是:
\[
\Delta h = \frac{N\lambda}{2}
\]

\insertfig{2-10.png}{劈尖条纹可能的移动示意图}{0.4}

下面来看牛顿环。其装置如图所示。一个曲率半径较大的透镜,放置于一玻璃平板上,便形成一空气薄膜,其等厚线是一系列以接触点为中心的圆,干涉图样如图。

\begin{figure}[htbp]
    \centering
    % 第一张图
    \begin{minipage}[b]{0.45\textwidth}
        \centering
        \includegraphics[width=0.55\textwidth]{2-11.png} % figure/image1.png
        \caption{牛顿环装置图}
    \end{minipage}
    \hfill
    % 第二张图
    \begin{minipage}[b]{0.45\textwidth}
        \centering
        \includegraphics[width=0.55\textwidth]{2-12.png} % figure/image2.png
        \caption{牛顿环干涉图样}
    \end{minipage}
\end{figure}

若中心点与平板玻璃密接,则中心点膜厚度为$0$,中心为零级暗纹。由此向外计算,设第$k$级暗环的半径为$r_k$,对应膜厚为$h_k$,则满足:
\[
2nh_k = k\lambda_0
\]
几何关系有:
\[
R^2 = r_k^2+(R-h)^2 \Rightarrow r^k = (2R-h_k)h_k \approx 2Rh_k
\]
因此得到:
\[
r_k = \sqrt{kR\lambda}
\]
这样,通过测量某一环的半径$r_k$和由其向外数第$m$环的半径为$r_{k+m}$,据此,可以算出透镜曲率半径:
\[
R = \frac{r_{k+m}^2-r_k^2}{m\lambda} = \frac{d_{k+m}^2-d_k^2}{4m\lambda}
\]

\insertfig{2-13.png}{利用牛顿环的变动指导透镜研磨}{0.4}

在光学冷加工车间,常常利用牛顿环及其变动来快速检测透镜表面曲率是否合格以及下一步应该如何研磨。如图,标准件$G$覆盖于待测工件$L$上,形成空气膜,出现牛顿环。圈数越多说明偏差越大,当人们说某工件表面公差为一个光圈时,就表示该工件与标准件之间的最大差距为$\frac{\lambda}{2}$.有经验的工人只要将标准件下压,根据光圈的伸缩情况就可以判断进一步研磨的位置。若光圈扩大(吐出),说明相同厚度的位置外扩,应该研磨中间;反之若光圈收缩(缩进),应该研磨两侧。

\subsection{等倾干涉}

现在我们研究在膜层厚度均匀,点光源照明条件下无穷远处的干涉场。

\insertfig{2-14.png}{等倾干涉装置图及干涉图样}{0.8}

如图所示,由点光源发出的一个光线,经过此薄层上下表面反射,恰好分为彼此平行的两条光线,而相干于无穷远。为了便于观测,也为了能获取完整的干涉条纹,实验上是将点光源设置在旁侧,经由一块倾斜的半反射镜,向下照射薄膜,并用一个透镜接收,将无穷远处的干涉场聚焦于其后焦面,呈现出干涉圆环图样。

\insertfig{2-15.png}{等倾干涉光程差}{0.5}

等倾干涉的光程差可以套用前面的结论:
\[
\Delta L = 2nh\cos i
\]
这里$i$是与入射角$i_1$对应的膜内的折射角,由于膜层光学厚度均匀,因此光程差唯一地决定于倾角$i$.这就是说等倾角的场点轨迹就是条纹的形状,显然其在后焦面上表现为一个圆环。

考虑第$k$级暗纹,我们有光程差公式:
\[
\Delta L = 2nh\cos i = k\lambda
\]
在角度变化不大时,我们可以在两边取差值:
\[
-2nh\sin i \Delta i = \Delta k \lambda \Rightarrow \Delta i = -\frac{\Delta k \lambda}{2nh\sin i}
\]
上式表明,$i_k$越大,$\Delta i$越小,倾角不大的时候,可以近似地认为相邻条纹半径之差正比于倾角之差。因此我们就说在干涉图样中离中心较远的地方的等倾条纹更加密集。

由前面的光程差公式可以知道,干涉图样中心处$i\approx 0$,$\cos i \approx 1$,光程差最大,干涉级次最高。由内而外干涉级次依次降低。当我们减小膜的厚度时,追踪某一固定级次的条纹,发现为了达到相同的光程差,必须相应地减小折射角,因此该级次的圆环由外向内移动到折射角较小的地方,从而在宏观上表现为干涉图样向内收缩,中心处不断有环吞入,中心环的干涉级次减小。反之膜的厚度增大时表现为干涉图样向外扩张,中心处不断有环吐出,中心环的干涉级次增大。

下面来看等倾干涉的最重要应用之一,迈克耳孙干涉仪。

\insertfig{2-16.png}{迈克耳孙干涉仪结构}{0.35}

如图所示,$M_1$,$M_2$是一对精密磨光的平面镜,$G_1$,$G_2$是完全相同的两块玻璃板,其中$G_1$的背面镀了一层很薄的银膜,使得从光源入射的光线在这里被分为为强度几乎相等的两部分,反射光束射向$M_1$, 被反射回来后再次透过$G_1$到达观测者,透射光束射向$M_2$, 被反射回来到达$G_1$镀银面再次反射到达观测者。这样的两束光相干产生干涉图样。$G_2$的意义在于补偿反射光比投射光多走过的光程。

现在我们分析迈克耳孙干涉仪产生的干涉图样。干涉仪产生的图样可以被等效为$M_1$与$M_2'$镜面形成的空气层反射所产生的干涉场。这里$M_2'$是右方$M_2$镜面对$G_1$镀银面反射而生成的像,因此整体的干涉可以等效于空气膜形成的薄膜干涉。具体的干涉图样我们如下给出,具体的分析留给读者完成。

\insertfig{2-18.png}{干涉仪中的干涉图样}{0.5}


\insertfig{2-17.png}{产生对应干涉图样的等效空气膜}{0.5}

迈克耳孙干涉仪的一个重要应用是精密测长。当$M_2$在空间中平移长度为$l$时,光程改变量为$2nl$,设某处干涉强度变化了$N$次(吞入或吐出$N$个条纹),则有:
\[
N \lambda_0 = 2nl \Rightarrow l = \frac{N\lambda}{2}
\]

另一个重要应用是测量双线的波长差。设用于照明的光源由两种波长为$\lambda_1$和$\lambda_2$的光合成的。假定在初始某位置$\lambda_1$的亮纹与$\lambda_2$的暗纹相重合,那么此时条纹最为模糊:
\[
N_0 \lambda_1 =(N_0-\frac{1}{2})\lambda_2
\]
增加光程差,直到下一次视场再次变得最为模糊时,有:
\[
3N_0 \lambda = 3(N_0-\frac{1}{2})\lambda_2, \quad \Delta L = 2N_0\lambda_1 = 2d
\]

其中$d$为两次模糊时空间中镜子移过的间距。根据前面两个式子,可以解出:
\[
N_0 = \frac{\lambda}{2\Delta \lambda}, \quad \Delta \lambda = \frac{\lambda^2}{2d}
\]

\section{光场的时间相干性}

在第一章中,我们曾经提到过定态波的概念。光源是一个由大量原子随机发射的光波列构成的集合。每个原子由激发态跃迁到基态时,会发出特定波长的光。由于更加精细的原子结构的影响,即便是同种原子也并不会都跃迁到相同的激发态,导致光源发出的光并不是严格单色的,而是具有一定的谱宽度,也即波长位于一个以$\lambda$为中心的区间内。

\insertfig{2-19.png}{单色光的实际物理图像}{0.5}

波列指的是一列长度有限的波的扰动。考虑光源每次发光的持续时间为$\Delta t$,则波列在真空中的长度为$\Delta x = c\Delta t$.而波列长度与波长谱的宽度有一反比关系,其可以被表示为:
\[
\Delta x \cdot \Delta \lambda = \lambda^2
\]
关于这式子的由来有兴趣的读者可以参阅本节末尾的注记。

下面我们来看看什么是时间相干性。考虑从同一光源在某一时刻发出的一个波列,其经过某种装置后被分为两列相干波列,这样被分开的两个波列之间是相干的,因为其具有恒定的相位差。只有这两列波在相近的时间内到达空间中的交叠光场,在我们考察的空间区域内才能发生干涉,否则看似相干的两束光,由于实际上来自于光源发出的不同波列,无法保证恒定的相位差,就观察不到干涉现象。这种因为光源的发光时间导致的干涉消失现象被称为光的时间相干性。

\insertfig{2-20.png}{波列长度与实际光程差的比较}{0.4}

如图所示设光程差为$\delta$,考虑可能发生干涉的两个光波列。$\delta=0$时,两波列在时间上完全重叠,干涉条纹的可见度最大。光程差小于波列长度但不为$0$,干涉条纹仍然可见。反之若光程差超过了波列长度,则完全不能产生干涉现象。迈克耳孙干涉仪中的补偿板就有一作用是为了防止两列波的光程差过大而无法被观测到相干现象。

能够发生干涉时允许的最大光程差被称为相干长度。这样根据我们前面给出的反比关系式$\Delta x \cdot \Delta \lambda = \lambda^2$,得到相干长度的表达式为:
\[
\delta_m = \frac{\lambda^2}{\Delta \lambda}
\]
我们可以发现时间相干性是与光谱的单色性强相关的,光谱的单色性越好,$\Delta \lambda$越小,相干长度越长,时间相干性越好。

\begin{noteenv}
    我们先引入光源发光的光强谱密度概念。我们采用波数来表征光的单色性。设光源发出的光中,在$k \sim k+\Delta k$中含有的光强是:
    \[
    \Delta I_0 = i(k)\cdot \Delta k
    \]
    取极限得到微分形式:
    \[
    \d I_0 = i(k)\d k, \quad i(k)= \frac{\d I_0}{\d k}
    \]
    这样$i(k)$就被称作入射光的光强谱密度函数,其物理意义是在波数$k$附近单位波数间隔内入射光中所含的光强。为了计算简明,我们采取一个简化模型,设光强的谱密度函数有如下的形式:
    \[
    i(k)= 
    \begin{cases}
    i_0, & |k-k_0|\le \frac{\Delta k}{2} \\
    0, & otherwise
    \end{cases}
    \]
    考虑谱元$k \sim k+\d k$.在这个谱元中各光束发生干涉,其干涉叠加强度为:
    \[
    \d I = \d I_0 (1+\cos k\Delta L)=i(k)(1+\cos k\Delta L)\d k
    \]
    总强度是各谱元贡献的非相干叠加,因为各谱元之间频率不同:
    \[
    I = \int_0^{\infty}i(k)(1+\cos k \Delta L)\d k
    \]
    同时注意到:
    \[
    \int_0^{\infty} i(k)\d k = I_0
    \]
    这是入射光总强度,于是有:
    \[
    I(\Delta L) = I_0+\int_0^{\infty}i(k)\cos (k\Delta L)\cdot \d k
    \]
    这是一个由任意的谱密度函数求解总光强的积分式,代入我们选择的谱函数,得到:
    \[
    I(\Delta L)=i_0\Delta k+i_0\int_{k_1}^{k_2}\cos (k \Delta L)\d k = I_0(1+\frac{\sin v}{v}\cos k_0 \Delta L)
    \]
    其中$v =\frac{\Delta k\Delta L}{2}$

    与一般的干涉公式对比,我们发现后面的干涉项受到了$\frac{\sin v}{v}$的调制,因此当$v = \pi$时干涉光强为$0$,此时有:
    \[
    \Delta k \cdot \frac{\Delta L}{2} = \pi \Rightarrow \Delta L_{max} = \frac{2\pi}{\Delta k}
    \]
    注意到波数与波长的关系:
    \[
    k = \frac{2\pi}{\lambda} \Rightarrow \Delta k = \frac{2\pi}{\lambda^2}\Delta \lambda
    \]
    代入上式,就得到:
    \[
    \Delta L_{max} = \frac{\lambda^2}{\Delta \lambda}
    \]
    这样我们得到了由单色性导出的最大光程差公式。根据时间相干性中的讨论,这个最大光程差应该与波列长度相等,于是有:
    \[
    \Delta x =\frac{\lambda^2}{\Delta \lambda} \Rightarrow \Delta x \cdot \Delta \lambda = \lambda^2
    \]
\end{noteenv}

\graphicspath{{figure/chap3/}}
\chapter{光的衍射}

\section{夫琅禾费单缝衍射}

\subsection{衍射现象概述}
当光波遇到障碍物,将或多或少地偏离几何光学的直线传播而绕行,这种现象统称为光的衍射.衍射使光强可以波及几何阴影区内,衍射也可以使几何照明区内出现暗纹或暗斑.总之,衍射效应使屏障以后的空间光强分布,既区别于几何光学给出的光强分布,又区别于光波自由传播时的光强分布,衍射光强有了一种重新分布.

衍射是一切波动均具有的传播行为.然而光波衍射却不易为人们所觉察,这有两点原因,一是可见光的波长极短,二是普通光源是非相干的面光源,当我们用一東高亮度的光束,照射各种形状且线度较小的开孔或屏障时,在较远的屏幕上,将呈现不同的衍射图样。

\insertfig{3-1.png}{各种衍射图样}{1}

衍射现象具有两个鲜明特点。其一是,当光束在衍射屏上的某一方位受到限制时,远处屏幕的衍射光强就沿着这个方向扩展开来,也就是说波具有顽强的反限制的特征。二是光孔限度与光波长之比是一个敏感因素,这一点可以用一个统一的公式来反映:
\[
\rho \Delta \theta \approx \lambda
\]

其中$\lambda$为光的波长,$\Delta \theta$为中心衍射斑的角宽度,$\rho$为狭缝的宽度。其中$\Delta \theta$越大,说明衍射效应越强,因此我们可以简单地说,狭缝宽度越小,衍射现象越明显,中心衍射斑铺展的越开。上面这个式子我们会在后面给出推导。

\insertfig{3-2.png}{用激光束观察衍射现象}{0.5}

\subsection{惠更斯-菲涅耳原理}

在正式阐述这一理论之前,我们先提请读者回忆在力学中曾经学习过的球面简谐波的波函数和复数表示法,它们对我们接下来的论述是有意义的。对于这部分不清楚的读者可以参阅大学物理力学部分的教科书,这里不再赘述。

对于点源在原点的球面波,其波函数可以被写为:
\[
U(r,t) = \frac{A}{r}\cos(\omega t-kr-\varphi_0)
\]
复数形式为:
\[
U(r,t) = \frac{A}{r}e^{ikr}\cdot e^{-i\omega t}
\]

\insertfig{3-3.png}{衍射积分式中各量的说明}{0.4}

下面阐述这原理的具体内容.用简短的文字概括起来,惠更斯-菲涅耳原理可表述如下:波前$\Sigma$上每个面元$\d \Sigma$都可以看成是新的振动中心,它们发出次波。在空间某一点$P$的振动是所有这些次波在该点的相干叠加。写成公式是:
\[
U(P) = \oiint_{\Sigma} \d U =  K \oiint_{\Sigma} f(\theta_0,\theta)U_0(Q)\frac{e^{ikr}}{r}\d S
\]
这公式中的各个符号可以被解释如下:$\d S$代表着微分面元; $U_0(Q)$是这些次波源本身的复振幅; $\frac{e^{ikr}}{r}$代表次波源发射球面波到达场点;$f(\theta_0,\theta)$被称为倾斜因子,用于表示次波源的发射并非各项同性。

基尔霍夫在大约这个公式提出的六十年后从定态波场的亥姆霍兹方程出发,利用矢量场论中的格林公式在$r\gg \lambda$条件下导出了无源空间边值定解的表达式:
\[
U(P) = \frac{-i}{\lambda} \oiint_{\Sigma} \frac{(\cos\theta_0+\cos\theta)}{2}U_0(Q)\frac{e^{ikr}}{r}\d S
\]

写成实数形式是:
\[
E(P,t) = \int_S \frac{E(Q)K(\varphi)}{r} \cos(\omega t-kr)\d S
\]

其中$P$为场点,$Q$为波前上一点,$K(\varphi)$为倾斜因子。

\insertfig{3-4.png}{衍射系统的分类}{0.8}

衍射系统以衍射屏幕为界,可以被分为前场和后场。在无成像的衍射系统中,通常按照光源,衍射屏,接收屏三者之间距离的远近分为菲涅耳衍射(图(a))和夫琅禾费衍射(图(b))。菲涅耳衍射是指,光源-衍射屏或者衍射屏-接收屏两者距离中至少有一个为有限远的衍射系统。夫琅禾费衍射则指的是衍射屏与光源和接收屏均是无限远的情形。简单而言,菲涅耳衍射是近场衍射,夫琅禾费衍射是远场衍射。

当然上面的分类也不尽准确,对于夫琅禾费衍射,我们常常使用透镜接收后场中的衍射光波并使其在透镜的后焦面上形成衍射光斑。

对于菲涅耳衍射,最著名的莫过于菲涅耳圆盘衍射产生的光斑。经过理论计算后菲涅耳指出,在使用一个遮光圆盘作为衍射屏时,接收面的中心会出现一个亮斑。这结论看起来不可思议,但是得到了实验证实。事实上,现在我们更多使用的还是夫琅禾费衍射,我们在下节内容中予以详细介绍。

\subsection{夫琅禾费单缝衍射}

\insertfig{3-5.png}{夫琅禾费衍射装置}{0.6}
夫琅禾费单缝衍射的装置如图所示.用平行光照射单缝,在透镜后焦面上接收衍射光波,单缝的宽度$a$远小于垂直纸面的长度$b$,这样衍射强度将显著地沿着$x$方向扩展。

为了求解衍射强度的分布,我们采用如下的方法:将缝面处的波阵面划分为$N$个与狭缝平行的等宽窄条,其中$N$非常大。注意到窄条发出的子波在$P$点的光振动产生的振幅近似相等,设为$E_0$。狭缝最边缘的两个窄条$A,B$发出的衍射光到达$P$点时的光程差$AC= a\sin\theta$。相邻的两个窄条发出的子波在$P$点的光程差为$\frac{a\sin\theta}{N}$,相位差为$\Delta \varphi =\frac{2\pi}{\lambda}\cdot \frac{a\sin\theta}{N}$.

因此点$P$处的光振动可以被视为$N$个同频率同振幅振动方向相同,相位依次相差一个恒量$\Delta \varphi=\frac{2\pi}{\lambda}\cdot\frac{a\sin\theta}{N}$.如图所示$N\to\infty$时,$N$个相接的折线将变为一个圆弧。

\begin{figure}[htbp]
    \centering
    % 第一张图
    \begin{minipage}[b]{0.48\textwidth}
        \centering
        \includegraphics[width=0.55\textwidth]{3-6.png} % figure/image1.png
        \caption{振幅矢量法合成振幅}
    \end{minipage}
    \hfill
    % 第二张图
    \begin{minipage}[b]{0.48\textwidth}
        \centering
        \includegraphics[width=0.55\textwidth]{3-7.png} % figure/image2.png
        \caption{$N \to \infty$时变为圆弧}
    \end{minipage}
\end{figure}

由几何关系:
\[
\beta = N\cdot \Delta \varphi= \frac{\pi a \sin\theta}{\lambda}, \quad E_p =2R\sin\beta, E_0 = 2R\beta\Rightarrow \frac{E_p}{E_0}=\frac{\sin \beta}{\beta}
\]

注意到光强与振幅的平方成正比,因此我们可以得到$P$点的光强:
\[
I = I_0(\frac{\sin\beta}{\beta})^2
\]

其中$I_0$为入射光强。

在这之后我们将多次遇见$\frac{\sin x}{x}$型函数,可以记作$\mathrm{sinc}(x)$.我们对这函数的一些性质在下面做些阐述。

\insertfig{3-8.png}{$\mathrm{sinc}(x)$图象}{0.65}

\insertfig{3-9.png}{$\mathrm{sinc}^2(x)$图象}{0.7}

(1)最大值(主极大). 当$x=0$,$\mathrm{sinc}(x)=\frac{\sin x}{x}=1$时为最大值.这在单缝衍射中,表现为$\theta=0$时,衍射强度$I(0)=I_0$为最大值,称其为零级衍射峰,其位置正是几何光学像点位置。这个位置也被称为单缝中央主极大光强,也被称为中央明纹中心。

(2)零点位置. $\mathrm{sinc}(x)$函数存在一系列零点.当$x=k\pi,k=\pm 1, \pm 2,\dots$时,$\mathrm{sinc}(x)=0$.这在单缝衍射中,表现为当:
\[
a\sin\theta = k\lambda , \quad k = \pm 1, \pm 2,\dots
\]
时,衍射强度$I(\theta)=0$,出现暗纹。上式称为单缝衍射零点条件。

(3)次极大. $\mathrm{sinc}$函数在相邻两个零点间存在一个极大值,这个位置由:$\frac{\d(\mathrm{sinc}(x))}{\d x} = 0 \Rightarrow x = \tan x$这个超越方程给出,其数值结果可以被列表如下:

\begin{table}[h!]
\centering
\renewcommand{\arraystretch}{1.8} % 行距
\setlength{\tabcolsep}{18pt}      % 列间距

\begin{tabular}{|c|c|c|c|}
\hline
$x$ 
& $\pm 1.43\pi$ 
& $\pm 2.46\pi$
& $\pm 3.47\pi$ \\
\hline
$\sin\theta$ 
& $\pm 1.43\,\frac{\lambda}{a}$ 
& $\pm 2.46\,\frac{\lambda}{a}$ 
& $\pm 3.47\,\frac{\lambda}{a}$ \\
\hline
\multirow{2}{*}{$\operatorname{\mathrm{sinc}}(x)$}
& 0.22 & 0.13 & 0.09 \\
\cline{2-4}
& $4.7\%$ & $1.7\%$ & $0.8\%$ \\
\hline
\end{tabular}

\end{table}

我们可以发现在次极大上的光强非常少,零级衍射斑集中了全部入射功率的$80\%$以上,因此我们常常用下面的半角宽度来表征衍射的强度。

(4)半角宽度. 零级衍射斑的角范围由零级衍射峰与其相邻暗点之间的角方位的差值予以度量,称其为零级衍射的半角宽度。$\Delta \theta_0 = \theta_1-\theta_0$. 在平行光正入射条件下,$\theta_0=0$,$\theta_1\approx \sin \theta_1 =\frac{\lambda}{a}$. 因此半角宽度为:
\[
\Delta \theta_0 = \frac{\lambda}{a} , \quad a \cdot \Delta \theta_0 = \lambda
\]

在进行计算的时候一定要注意这是半角宽度!
考虑到几何装置图,零级衍射斑在后焦面上铺展开的宽度是:
\[
\Delta x_0 = 2f\cdot \tan\theta_1 \approx 2f\Delta \theta_0 = 2\frac{f\lambda}{a}
\]

其他明纹的宽度大致可以被表示为:
\[
\Delta x \approx \frac{1}{2}\Delta x_0 = \frac{f\lambda}{a}
\]

衍射效应的强弱我们用零级衍射的半角宽度予以度量。半角宽度越大,说明衍射光强在$x$方向铺展的越开,衍射效应越明显。根据$\Delta \theta_0 = \frac{\lambda}{a}$,缝宽$a$越小,波长越大,半角宽度越大,衍射效应越明显。

下面我们用一道例题来说明不同波长的光叠加时的影响。

在单缝夫琅禾费衍射实验中,入射的平行光含有两种波长成分,红光$\lambda_1=600nm$,蓝光$\lambda_2=400nm$,且两者光强相等,试分析这二色光衍射图样的主要区别。

主要分别有两点.一是红光与蓝光各自展开的衍射半角宽度不同.
\[
\frac{\Delta \theta_{10}}{\Delta \theta_{20}}=\frac{\lambda_1}{\lambda_2} =1.5
\]
也就是说相对于蓝光,红光图样更加扩展。

二是红光与蓝光衍射斑中心强度不等:
\[
\frac{I_{10}}{I_{20}}=\frac{A_1^2/\lambda_1^2}{A_2^2/\lambda_2^2} = \frac{\lambda_2^2}{\lambda_1^2} = 45\%
\]
也就是红光衍射斑中心的强度几乎比蓝光小一半。同时可以想见,可能在某处出现红光和蓝光的次极大和暗纹交叠的情况。

当光源在垂直纸面方向进行扩展时,干涉图样也会随之进行扩展,因为在垂直纸面方向上并不发生衍射。

当光源上下移动时,零级主极大仍然位于几何像点的位置,因此条纹会整体向光源移动的反方向移动。

狭缝上下移动时,仅仅是改变了入射的光强,整体的条纹只有亮度的变化而没有位置的移动。

\subsection{单缝衍射对双缝干涉的调制}
不考虑衍射时,双缝干涉的光强分布为:
\[
I = 4I_0\cos^2(\frac{\pi d}{\sin\theta})
\]
其中双缝的间距$d=a+b$,其中$a$是缝宽,$b$是缝间不透光部分的宽度。

在考虑单缝衍射的影响后,在衍射屏上的叠加可以被看作是两个单缝衍射的光强在后焦面上相干叠加的结果。

\insertfig{3-10.png}{单缝衍射因子调制双缝干涉}{0.6}

这样的相干叠加显然是只改变光强的大小的,我们仅仅需要修改光强的表达式,使之与角度相关,也就是:
\[
I = 4I_0(\frac{\sin\beta}{\beta})^2\cos^2(\frac{\pi d\sin\theta}{\lambda})
\]

我们分别称$(\frac{\sin\beta}{\beta})^2$和$\cos^2(\frac{\pi d\sin\theta}{\lambda})$为单缝衍射因子和缝间干涉因子。缝间干涉因子变化较快,单缝衍射因子变化较慢。当缝间干涉因子满足明纹条件,但单缝衍射因子满足暗纹条件时,本应该出现明纹的地方会出现暗纹,这种现象被称为在单缝衍射因子的调制下双缝干涉出现了缺级。

发生缺级时,满足:
\[
d\sin\theta = k\lambda, \quad a\sin\theta = k'\lambda \quad \frac{d}{a} = \frac{k}{k'}
\]
时,将会出现缺级。由$k'=1,2,3\dots$可以依次推出缺级的级次为$\frac{d}{a},\frac{2d}{a},\frac{3d}{a}\dots$·    
\section{光栅衍射}
在上一节的最后,我们讨论了单缝衍射对于双缝干涉的调制,进一步地,我们将会考虑一种特殊的光学元件——光栅。光栅是一种由大量的等宽等间距的平行狭缝构成的光学元件。相邻缝对应点之间的距离为缝间距。由于其是一个描绘光栅性质的基本量,我们称其为光栅常数。光栅常数还常常用线数表示,例如$300$线光栅代表着该光栅$1mm$内有$300$条刻痕。

\insertfig{3-11.png}{光栅衍射示意图}{0.75}

现在我们来考虑光栅衍射时衍射屏上的光强分布。对于每个单缝,设其入射的振幅为$E_0$,则其在$\theta$衍射角方向上的光振幅为$E_{single}=E_0\frac{\sin \beta}{\beta}$,$\beta= \frac{2\pi}{\lambda}a\sin\theta$.相邻缝发出的光在$P$点的相位差为$\Delta \varphi= \frac{2\pi}{\lambda}d\sin\theta$.因此我们发现这类似于我们在处理单缝衍射的时候的方法,只不过这里不是无穷多缝的叠加。

\insertfig{3-12.png}{光栅衍射光强推导}{0.35}

如图所示,我们用振幅矢量法求解。由几何关系知:
\[
E_p = 2R\sin \frac{N\Delta\varphi}{2}, \quad E_{single} = 2R\sin\frac{\Delta \varphi}{2}
\]

这样我们可以定义一个新的宗量$\alpha=\frac{\Delta \varphi}{2}=\frac{\pi d}{\lambda}\sin\theta$,则最终的光强可以被写为:
\[
E_p = E_0 \frac{\sin \beta}{\beta}\frac{\sin N\alpha}{\sin \alpha} \Rightarrow I_p = I_0(\frac{\sin \beta}{\beta})^2(\frac{\sin N\alpha}{\sin \alpha})^2
\]

我们称$(\frac{\sin \beta}{\beta})^2$为单缝衍射因子,$(\frac{\sin N\alpha}{\sin \alpha})^2$为缝间干涉因子。

\insertfig{3-13.png}{光栅衍射光强分布示意图,$\frac{d}{a}=4$}{0.6}

我们对这光强表达式作一分析:当$\alpha=\pm k\pi,k\in \mathbb{N}$时,缝间干涉因子取到最大值:
\[
\lim_{\alpha \to k\pi} (\frac{\sin N\alpha}{\sin \alpha})^2 = N^2
\]
我们称此时的极大为主极大,由$\alpha = \frac{\pi d}{\lambda}\sin\theta = \pm k \pi$,得到:
\[
d\sin\theta = k\lambda
\]
上面这个方程被称为光栅方程,它给出了各级主极大的角位置。

而当$\sin N\alpha = 0$,$\sin\alpha\neq 0$时,$I_p=0$.此时光强极小。
因此有:
\[
N\alpha = \pm k' \pi \qquad k' = 1,2,\cdots \neq Nk \Rightarrow d\sin\theta= \pm\frac{k'}{N}\lambda
\]

因此每个主极大之间有$N-1$个极小,而每个极小之间又有一个次极大,因此每个主极大之间有$N-2$个次极大。

主极大的半角宽度$\Delta \theta_k$应该根据两侧的第一个极小的位置给出,根据前面的分析,容易得到此时$k' = \frac{Nk\pm1}{N}$:
\[
d\sin\theta_k  = k\lambda, \quad d\sin(\theta_k\pm\Delta \theta) = (k\pm\frac{1}{N})\lambda
\]
由以上两式,做小量近似之后得到:
\[
d\cos\theta_k \Delta \theta_k = \frac{\lambda}{N}
\]
从而得到$k$级主极大的半角宽度为:
\[
\Delta \theta_k = \frac{\lambda}{Nd\cos\theta_k}
\]
以上我们主要分析了缝间干涉因子的影响,但是由于单缝衍射因子的调制,可能会出现某个主极大我们观测不到的情况,这种现象被称为光栅衍射的缺级。

单缝衍射的暗纹方程是:
\[
a\sin\theta = \pm k'' \lambda 
\]
光栅衍射主极大由光栅方程给出:
\[
d\sin\theta = \pm k \lambda
\]
因此当:
\[
\frac{d}{a} = \frac{k}{k''}
\]
时,干涉明纹与衍射暗纹相叠加,屏幕上表现为暗纹,这被称为干涉明纹的缺级。

干涉明纹缺级的级次可以被表示为:
\[
k = \frac{d}{a}k'', k'' = 1,2,3\dots
\]

\insertfig{3-16.png}{晶体面间干涉的布拉格条件}{0.6}

以上我们研究的都是一维的光栅,在X射线被发现后,人们一直不知道其有何种用途。X射线的波长非常短,只有埃量级,因此若欲想使其通过光栅发生衍射,大约需要$10^5$线光栅,这显然已经远超人类的机械精加工极限。但是晶体的晶格常数恰巧处于合适的量级,可以发生衍射。我们仅对晶体衍射做一简单介绍。

如图所示,我们用不同方向的X射线照射晶体,由于晶体具有有序结构,其可以被分为一层层的晶片。设晶片间的间隔为$d$,则计算上下两光线的光程差:
\[
\Delta L=2d\sin\theta 
\]
这可以被看作是缝间干涉因子,因此当$2d\sin\theta = k\lambda$时,干涉加强,衍射达到极大。上述条件被称为布拉格条件,是晶体衍射学的基本公式。

\section{光学仪器的分辨本领}

\insertfig{3-14.png}{圆孔夫琅禾费衍射与艾里斑}{0.8}

当我们将衍射孔的形状设置为一个圆孔时,衍射屏上的衍射图样如图所示,中心的亮斑被称为艾里斑,其集中了全部光强的$80\%$以上,经过数学计算,其第一个暗环的角方位$\theta_{1}$满足:
\[
\frac{\pi D \sin\theta_1}{\lambda} = 1.22\pi
\]
因此艾里斑的半角宽度为:
\[
\theta_1 = 1.22\frac{\lambda}{d}
\]

由于镜头对波前的限制而产生的衍射效应,使物点发射的光波在像面上不可能形成一个像点,而是以像点为中心扩展为一定的强度分布,集中了大部分光功率的中心光斑,就是圆孔夫琅禾费衍射的艾里斑.这就是说,即使不计较镜头的所有几何像差,成像光学仪器也无法实现点物成点像的理想情况.因此,物面上相距很近的两个物点,反映在像面上就是两个可能重叠的衍射斑,这两个衍射斑甚至可能过度重叠,变成模糊一团,以致观察者无法辨认物方两个物点的存在.总之,物方图像是大量物点的集合,而变换到像面上的强度分布却是大量艾里斑的集合,它不可能准确地反映物面上的所有细节。

成像仪器分辨细节能力的定量表示称作分辨本领,也称为分辨率或分辨力.这首先涉及一个是否可分辨的标准问题。通常我们对此采取瑞利判据。

考虑两个物点,其反映在像面上有两个艾里斑,设这两个艾里斑的中心的角间隔为$\delta \theta$,每个艾里斑的半角宽度为$\Delta \theta_0$,那么瑞利判据认为:
\begin{center}
    $\delta \theta > \Delta \theta_0$时,可以分辨; $\delta \theta < \Delta \theta_0$时,不可分辨; $\delta \theta = \Delta \theta_0$时恰好可以分辨。
\end{center}

这就是说瑞利判据规定,当一个像的艾里斑中心恰好落在另一个像的边缘暗环的时候,认为这两个像斑恰好可以分辨。

光学仪器的角分辨本领被定义为最小分辨角的倒数。根据瑞利判据,最小分辨角恰为艾里斑的半角宽度,也就是:
\[
\delta \theta = \theta_1 \approx 1.22\frac{\lambda}{d}
\]
因此角分辨本领为:
\[
R = \frac{1}{\delta \theta}  = \frac{d}{1.22\lambda}
\]

上一节中介绍的光栅,其作为一种光学仪器,主要用途之一就是制成光谱仪。光谱仪是一种能够将不同波长的光区分开的仪器。为了描述光谱仪分光能力的强弱,我们定义角色散本领为单位波长间隔的两条谱线散开的角度大小:
\[
D_\theta = \frac{\d \theta}{\d \lambda}
\]
在光栅中,我们用于分光的是各级别的主极大(除0级):
\[
d\sin\theta = k \lambda \Rightarrow d\cos\theta\d\theta = k \d \lambda \Rightarrow D =\frac{k}{d \cos\theta}
\]

考虑到每个主极大有自身的半角宽度,我们之前已经求出:
\[
\delta \theta_k = \frac{\lambda}{Nd\cos\theta_k}
\]

因此角间隔为$\delta \theta$两个主极大就会发生重叠,根据瑞利判据,我们令:
\[
\delta \theta = D_\theta = \frac{k}{d\cos\theta_k}\delta \lambda, \quad \Delta \theta_k = \frac{\lambda}{Nd \cos\theta_k}
\]
两者相等。前者是我们根据角色散本领推导出的两主极大之间的角差距,后者是主极大本身的半角宽度,瑞利判据告诉我们两者相等的时候恰好无法分辨。于是我们得到:
\[
\delta \lambda_m = \frac{\lambda}{kN}
\]
光栅的色分辨本领描绘的是光栅对波长相近的非单色光的区分能力,其被定义为:
\[
R_\lambda = \frac{\lambda}{\delta \lambda_m}=kN
\]


\graphicspath{{figure/chap4/}}
\chapter{光的偏振}
\section{偏振光概述}
\insertfig{4-1.png}{光的偏振示意图}{0.7}
作为特定波段的电磁波,光波自然是一种横波。在光与物质相互作用的过程中,主要是光波中的电场矢量起作用,因此把电场矢量称之为光矢量,其所在的平面与光射线方向正交。在这个振动平面上光矢量的振动方向对于光的传播方向是不具有轴对称性的,因此这种不对称现象就被称为偏振。纵波的振动方向与波的传播方向一致,从垂直于波传播方向的各个方向去观察纵波都是一致的,因此偏振是横波区别于纵波的重要标志。

在垂直于光波的传播方向的平面内,光矢量可能有不同的振动状态,每种振动状态都可以被称为一个偏振态,最常见的光的偏振态大致有五种,我们一一予以介绍。

\insertfig{4-2.png}{光的五种宏观偏振态}{1}

(1)线偏振光。 在观测时间内光矢量的方向始终不变,如图(a)所示,其振动可以被表示为:
\[
\bm{E}(t) = \bm{A} \cos\omega t \quad \text{or} \quad E_x(t)=A_x\cos\omega t, \; E_y(t) = \cos(\omega t+\delta)
\]

两个正交振动的相位差$\delta$决定了光的振动象限,振幅比$\frac{A_y}{A_x}$决定了光矢量的线偏振角。

(2)自然光。 自然光是由大量的不同取向的彼此无关的特殊优越取向的线偏振光构成的。因此自然光相对传播方向室友轴对称性的,如图(b)所示。各种普通光源,如日光,钠光灯等均属自然光。这是因为虽然这些光源在微观持续发光时间$\tau_0$内发射的一段波列是线偏振光,但是其频率,相位,振动方向,波列长度等均不同,因此不存在哪个方向相对其他更具优势。

(3)部分(平面)偏振光或称部分线偏振光,其与自然光的差距仅在于其不具有轴对称性而是具有一个优越方向,如图(c)所示。在优越方向上其具有最大的光强。自然光经界面反射和折射后一部分将会变为部分偏振光。我们用偏振度描述其偏振程度,设部分偏振光最大振幅对应的光强为$I_{\max}$,最小振幅对应的光强为$I_{\min}$,则我们定义偏振度:
\[
P = \frac{I_{\max}-I_{\min}}{I_{\max}+I_{\min}}
\]

当偏振度为$0$时,部分偏振光退化为自然光:当偏振度为$1$时,部分偏振光退化为线偏振光。

圆偏振光和椭圆偏振光我们会在后面介绍。

下面我们介绍线偏振光的获得。

\insertfig{4-3.png}{偏振片与透振方向}{0.7}

在一张硝化纤维薄膜上敷上一层硫酸碘奎宁超微晶粒,在拉伸之后,这些晶粒定向有序排列而固化于这张薄膜上,形成一张人造偏振片。偏振片对于入射光的电矢量的吸收有强烈的方向性,一个方向可以充分地透射,另一个与其正交的方向上电矢量被强烈地吸收。我们分别称这两个方向为透振方向和消光方向。一般而言拉伸方向是消光方向,透振方向与拉伸方向垂直。

任何一种偏振态的光束,经过偏振片后都会变成线偏振光。凡是能够产生线偏振光的期间都被称为起偏振器,简称起偏器。用于检验线偏振光的偏振片或者其他起偏器都被称为检偏振器,简称检偏器。

如前图所示,我们考虑一个线偏振光透过偏振片。设线偏振光原有的偏振方向为$A_0$,其中与透振方向一致的平行分量$A_{\parallel}$可以透过偏振片,而垂直分量$A_{\perp}$被$P$吸收,因此透射光强$I_P=A_{\parallel}^2 = A_0^2\cos^2\alpha$. 也就是说:
\[
I_p(\alpha) = I_0\cos^2\alpha
\]
其中$I_0$为入射光强,$\alpha$为入射线偏振光的偏振方向与偏振片透振方向的夹角,这式子被称为马吕斯定律。注意这定律仅适用于线偏振光。马吕斯定律没有考虑偏振片对于光的吸收,如果设偏振片对于光强的透过率为$\gamma$,则我们可以给出:
\[
I = \gamma I_0 \cos^2 \alpha
\]

据此,当一个偏振片$P$,面对一束入射的线偏振光而旋转时,$\alpha$历经$0 \to 2\pi$的变化,其光强会出现: 最亮$\to$消光$\to$最亮$\to$消光,彼此之间间隔为$\frac{\pi}{2}$.若以一个偏振片作为检偏器,只有线偏振光在旋转过程中会出现消光的现象。

接下来我们考察自然光和部分偏振光透过偏振片的影响。对于自然光,无论偏振片位于何处,对自然光来说是没有去别的,其光强按照马吕斯定律以$\cos^2\theta$比率通过,注意到在一个周期内$\cos^2\theta$的均值为$\frac{1}{2}$,因此透射光强为入射光强的一半。

\insertfig{4-4.png}{部分偏振光透过检偏器}{0.6}

对于部分偏振光,当偏振片作为检偏器旋转一周时,不会出现消光现象,而是出现一个极小光强。我们以极大$I_M$和极小光强$I_m$出现时的透振方向构建正交坐标架。我们可以把构成部分偏振光的大量不同取向的线偏振光分解为两个正交振动,其光强分别为$I_x=I_m$,$I_y=I_M$,入射光的总光强为$I_0=I_x+I_y=I_m+I_M$,因为这两个正交振动是彼此完全不相干的。因此其他方向的透射光强$I_p(\alpha)$就等于$I_m$,$I_M$按照马吕斯定律在$\alpha$方向贡献的和:
\[
I_p(\alpha) = I_m\cos^2\alpha+I_M\sin^2\alpha= I_m(1-sin^2\alpha)+I_M\sin^2\alpha = I_m+(I_M-I_m)\cos^2\beta
\]
根据这个光强形式,我们可以看出其中的第一项类似于自然光透过偏振片的光强,第二项类似于线偏振光透过偏振片的光强,因此我们可以说部分偏振光是由光强为$2I_m$的自然光和$I_M-I_m$的线偏振光混合而成的,线偏振光的偏振方向沿着出现$I_M$的透振方向。

综合以上的讨论,我们知道了偏振片能够区分自然光,线偏振光和部分偏振光。其透过旋转一周的检偏器时出现的现象分别是光强不变,出现消光现象和出现极小但不为$0$的光强三种。

接下来我们将讨论光在介质界面上发生反射和折射时的偏振现象,此部分的理论超出了课程的要求,我们以注记的形式给出一部分,有兴趣的读者可以自行查阅相关资料。
\begin{noteenv}
    \insertpic{4-5.png}{光在介质界面反射折射的基本图像}{1}

    在这张图中$\bm{k_1},\bm{k_1'},\bm{k_2}$分别表示入射光,反射光,透射光的波矢,$\bm{E_1}$为入射光的电矢量,其被分解为平行入射面的$\bm{E_p}$和垂直入射面$\bm{E_s}$两个正交分量,分别称为$p$振动和$s$振动,也把对应偏振方向的光称为$p$光和$s$光。当入射光只有$p$光时,反射和透射光也只有$p$光,$s$光同理,这也说明了$p$和$s$两个振动确实是彼此独立的。

    历史上,菲涅耳导出了发生反射和透射时的复振幅反射率和透射率的公式:

\[
\left\{
\begin{aligned}
\tilde{r}_{\mathrm{p}} &= \frac{\tilde{E}'_{1\mathrm{p}}}{\tilde{E}_{1\mathrm{p}}} = \frac{n_2 \cos i_1 - n_1 \cos i_2}{n_2 \cos i_1 + n_1 \cos i_2}= \frac{\tan(i_1 - i_2)}{\tan(i_1 + i_2)}, \\[8pt]
\tilde{r}_{\mathrm{s}} &= \frac{\tilde{E}'_{1\mathrm{s}}}{\tilde{E}_{1\mathrm{s}}} = \frac{n_1 \cos i_1 - n_2 \cos i_2}{n_1 \cos i_1 + n_2 \cos i_2} = \frac{\sin(i_2 - i_1)}{\sin(i_2 + i_1)};
\end{aligned}
\right.
\]

\[
\left\{
\begin{aligned}
\tilde{t}_{\mathrm{p}} &= \frac{\tilde{E}_{2\mathrm{p}}}{\tilde{E}_{1\mathrm{p}}} = \frac{2n_1 \cos i_1}{n_2 \cos i_1 + n_1 \cos i_2}, \\[8pt]
\tilde{t}_{\mathrm{s}} &= \frac{\tilde{E}_{2\mathrm{s}}}{\tilde{E}_{1\mathrm{s}}} = \frac{2n_1 \cos i_1}{n_1 \cos i_1 + n_2 \cos i_2}= \frac{2 \cos i_1 \sin i_2}{\sin(i_2 + i_1)}.
\end{aligned}
\right.
\]

这里的复振幅反射率可能出现负数,其符号代表着半波损失的出现,我们也可以自然地看出,当$n_2>n_1$时,$i_2>i_1$,此时出现了半波损失,反之则没有,且透射光无论何时都不会出现半波损失。

在电磁学的学习中,我们已经了解到了对于平面电磁波,我们的光强有如下公式:
\[
I = \langle S \rangle = \frac{1}{2}\sqrt{\frac{\varepsilon}{\mu}E_0^2}
\]
对于光频段,$\mu\approx \mu_0$,而在第一章中我们已经说明$n=\sqrt{\varepsilon_r}$,因此电磁波的强度可以被表为:
\[
I \approx \frac{n}{2c\mu_0}E_0^2
\]
因此光强的反射率和透射率可以被表示为:
\[
R_p = r_p^2 , \quad R_s = r_s^2 , \quad T_p = \frac{n_2}{n_1}t_p^2, \quad T_s = \frac{n_2}{n_1}t_s^2
\]

根据上述光强反射率公式,我们可以给出这样的一个曲线,在给定折射率的情况下描绘处光强反射率随着入射角的变化曲线。

\insertpic{4-6.png}{光强反射率曲线}{0.8}

由图示可以发现,反射光束中$s$光的光强反射率随着入射角增大单调上升,而$p$光的光强反射率先下降到$0$,而后很快地上升。这个使得$R_p$为$0$的角度就被称为布儒斯特角。

根据我们之前给出的公式, $i_1+i_2=\frac{\pi}{2}$时$p$光的光强反射率为$0$,结合折射定律,我们解出:
\[
\tan i_B  =\frac{n_2}{n_1}
\]
\end{noteenv}
在入射光以布儒斯特角$\tan i_B  =\frac{n_2}{n_1}$入射时,反射光线恰好与折射光线成一个直角。此时的反射透射情况可以被总结为:反射的线偏振光强度很小,折射的部分偏振光的偏振度很低。反射光的强度只有入射光的$15\%$,而经过一次透射的部分偏振光的偏振度只有$8\%$,因此人们往往采用多个平行玻璃板叠加的方式,获得偏振度较好的透射偏振光。

\begin{figure}[htbp]
    \centering
    % 第一张图
    \begin{minipage}[b]{0.35\textwidth}
        \centering
        \includegraphics[width=\textwidth]{4-7.png} % figure/image1.png
        \caption{布儒斯特角入射时反射光与折射光正交}
    \end{minipage}
    \hfill
    % 第二张图
    \begin{minipage}[b]{0.64\textwidth}
        \centering
        \includegraphics[width=\textwidth]{4-8.png} % figure/image2.png
        \caption{玻片组获取透射偏振光}
    \end{minipage}
\end{figure}


\section{双折射}
晶体是一种各向异性的介质,固体物理学的晶格几何理论表明,所有的晶体可以被分为七种晶系,按照光学性质,大致可以分为三种.

(1)单轴晶体. 三角晶系,四角晶系,六角晶系均系此类。如方解石,红宝石,石英,冰等等,常常提到的冰洲石属方解石的一种,分子式为$CaCO_3$.其晶胞呈平行六面体形.

(2)双轴晶体. 单斜晶系,三斜晶系,正交晶系均系此类。如蓝宝石,云母等。

(3)立方晶系. 如$NaCl$晶粒。它系各向同性介质。

双折射的现象如下所述。一细光束正入射于一块冰洲石表面,将有两束光透射出来,其中一束光沿着入射光方向直接透射,遵从通常的折射定律,另一束光从另一高度透射,这意味着这束光在冰洲石内发生了偏折,其不遵从通常的折射定律,这一现象被称为双折射。遵从通常各向同性介质中折射定律的光称为寻常(ordinary)光,简称为o光; 不遵从通常折射定律的光,称为非常(extraordinary)光,简称为e光。

\insertfig{4-9.png}{双折射现象}{0.45}

若我们的入射光是自然光,还能进一步发现,出射的两束透射光是线偏振光,且偏振方向不相同。当用一个偏振片去观察两光束时,一个清晰一个消失,转过$\frac{\pi}{2}$后恰好反转。

\insertfig{4-10.png}{方解石晶胞与光轴}{0.3}

晶体中存在一个特殊的方向,光沿着这个方向传播时不发生双折射现象,这个方向称为晶体的光轴。冰洲石晶体的光轴方向沿着其两个顿棱角顶点的连线方向。需注意光轴不是一条线,而是晶体中的一个特殊方向。

据此,我们可以引入若干平面。设一光线以$\bm{r}$入射于晶体表面,对于一个入射点,有两个方向值得注意:晶体表面的法线$\bm{N_s}$和晶体内部的光轴$\bm{z}$.由$\bm{(N_s,z)}$组成的平面被称为晶体主截面,此外还有一个入射面,其是由$\bm{(N_s,r_1)}$组成的平面。实验上法线,当入射面与主截面重合时,e光的偏折仍然在入射面内,而当入射面与主截面不一致时,e光射线就可能不在入射面内。同时我们定义主平面$\bm{(r,z)}$,其是由光线和光轴组成的平面。

上面我们介绍了晶体双折射现象。惠更斯在其著作中提出了一个理论模型,下面予以介绍。

\insertfig{4-11.png}{双折射的理论解释}{1}

设想单轴晶体内有一个点光源,如图所示,其中$\bm{z}$轴方向为晶体光轴方向,用一系列平行的虚线表示。沿着任意传播方向$\bm{r}$考察光波的传播行为,对o光和e光应该分别考虑,我们定义o光的振动光矢量$\bm{E_o}\perp\text{主平面}\bm{(r,z)}$, e光的振动光矢量$\bm{E_e}\parallel\text{主平面}\bm{(r,z)}$.

o光的振动的传播具有各向同性,其波前为球面$\sigma_o$,传播速度为$v_o$,与传播方向角$\xi$无关。e光的振动传播具有各向异性,e光的传播具有各向异性,其波前为旋转椭球面,转轴为光轴$\bm{z}$.两者的波面相切于光轴,e光的传播速度$v$与方向角$\xi$有关,其形式为:
\[
v^2 = \frac{v_e^2v_o^2}{v_e^2\cos^2\xi+v_o^2\sin^2\xi}
\]
这公式不需要读者掌握,但我们可以看出$\xi=0$时,其传播速度为$v_o$,也说明了沿着光轴方向上传播不出现双折射现象; $\xi=\frac{\pi}{2}$,其传播速度为$v_e$。我们可以引入折射率$n$来表示光速,与上述$v_o$,$v_e$对应的折射率$n_o$,$n_e$被称为单轴晶体的两个主折射率,即为:
\[
n_o = \frac{c}{v_o}, \quad n_e =\frac{c}{v_e}
\]
对于其他传播方向,折射率可以被表示为:
\[
n^2 = n_o^2\cos^2\xi+n_e^2\sin^2\xi
\]
可以看出对于其他传播方向,e光传播速度介于$v_o,v_e$之间,折射率也介于$n_o,n_e$之间。

根据$v_e>v_o$或者$v_e<v_o$,可以将晶体划为正负两类。对于负晶体:
\[
v_e \geq v_e(\xi) \geq v_o , \quad n_e \leq n_e(\xi) \leq n_o
\]
对于正晶体
\[
v_e \leq v_e(\xi) \leq v_o , \quad n_e \geq n_e(\xi) \geq n_o
\]

\insertfig{4-12.png}{正负晶体的波面示意图}{0.6}

简而言之,对于负晶体,e光为快光,o光为慢光。反之对于正晶体,o光为快光,e光为慢光。

如果我们关注光矢量与$\bm{z}$的夹角,我们可以更深刻地理解e光传播的各向异性。由于e光振动平行于主平面,故对于不同的传播方向$\xi$,其光矢量与光轴的夹角$\beta=\frac{\pi}{2}-\xi$也随之改变,$\beta$在$0$到$\frac{\pi}{2}$间取值,于是e光速度也介于$v_e,v_o$之间。而o光的光矢量垂直于主平面,故对于任意的传播方向角,其与光轴的夹角始终为$\frac{\pi}{2}$,因此o光的传播具有各向同性。且在沿着光轴方向时,两者的光矢量与光轴夹角均为$\frac{\pi}{2}$,故不出现双折射现象。

若我们知道光在介质或者晶体中的传播速度,就可以通过惠更斯作图法求得折射光线的方向。晶体中的惠更斯作图法与我们在第一章第一节中介绍的惠更斯作图法几乎一致,其区别仅在于界面上的一个点次波源将产生两个次波面进入晶体,一个是o光次波面,呈球面;另一个是e光次波面,呈旋转椭球面。相应地会产生两个包络面$\Sigma_o$,$\Sigma_e$,分别为o光和e光的宏观波面。下图展示了我们通过惠更斯作图法确定e光射线$\bm{r_e}$和光矢量方向$\bm{E_e}$和o光射线$\bm{r_o}$和光矢量方向$\bm{E_o}$。这里我们设定了入射面和晶体主截面重合于纸面,从而各折射光线均位于入射面内。

\insertfig{4-13.png}{晶体中的惠更斯作图法}{1}

我们可以从中得到两个重要结论。

(1)o光射线满足折射定律,e光射线不满足折射定律,甚至可能出现图4.14中的怪异偏折。仅当光轴垂直入射面的时候,o光与e光均满足折射定律。

(2)o光射线$\mb{r_o}$总是正交于其波面$\Sigma_o$,而e光射线$\mb{r_e}$与波面$\Sigma_e$并不一定正交。

\insertfig{4-14.png}{两个结论示意图}{0.6}

我们再考虑两种重要的入射情形。

(1)晶片厚度均匀,光轴平行表面且光束正入射。此时不难分析得知,o光和e光的波面平行于晶体表面,且光射线方向$\mb{r_o}$和$\mb{r_e}$均与波面正交,而从晶体另一表面正出射。表观上看并无双折射,但由于两者在晶体内的传播速度不同,从而导致两者经过该晶片厚度$d$后,两者光程不同,从而使得出射的两个正交光振动之间出现了相位差:
\[
\delta = \frac{2\pi}{\lambda}d(n_e-n_o)
\]
这被应用于制作波晶片,这是一种检验圆偏振光或椭圆偏振光的元件,留待后面介绍。

\insertfig{4-15.png}{光轴平行表面且正入射}{0.26}

(2)晶片厚度均匀,光轴任意取向且光束正入射。虽然光轴任意取向,简单分析可以确定此时e光波面仍然平行于晶体表面。此时体内的e光射线方向$\bm{r_e}$也是倾斜的,与波面法线方向$\bm{N_e}$并不一致,而是存在一个分离角$\alpha$:
\[
\alpha = \xi - \theta
\]
这里$\xi$角是传播方向与光轴$\mb{z}$的夹角,$\theta$角指的是波面法线$\mb{N_e}$与光轴$\mb{z}$的夹角。

\insertfig{4-16.png}{光轴任意且正入射}{0.72}

总结一下,我们在单轴晶体的双折射中,对于一个入射点我们需要关注的有:

6 个方向:入射光线方向 $\boldsymbol{r}_1$,表面法线方向 $\boldsymbol{N}_s$,晶体光轴方向 $\boldsymbol{z}$,体内 o 光射线方向 $\boldsymbol{r}_o$,体内 e 光射线方向 $\boldsymbol{r}_e$,体内 e 光波面 $\Sigma_e$ 法线方向 $\boldsymbol{N}_e$;

4 个面:入射面 $(\boldsymbol{r}_1, \boldsymbol{N}_s)$,晶体主截面 $(\boldsymbol{N}_s, \boldsymbol{z})$,o 光主平面 $(\boldsymbol{r}_o, \boldsymbol{z})$,e 光主平面 $(\boldsymbol{r}_e, \boldsymbol{z})$;

3 个角:$\boldsymbol{r}_e$ 与光轴 $\boldsymbol{z}$ 之夹角 $\xi$,$\boldsymbol{N}_e$ 与光轴 $\boldsymbol{z}$ 之夹角 $\theta$,$\boldsymbol{r}_e$ 与 $\boldsymbol{N}_e$ 之夹角 $\alpha$.

\insertfig{4-17.png}{双折射现象总结}{0.4}

下面我们介绍若干基于双折射现象制作的光学器件。首先是晶体棱镜。(下面的图中都用虚线标示光轴方向)

晶体棱镜通常是由两块按照一定方式切割下来的晶体三棱镜组合而成。通过晶体棱镜,入射的自然光被分解为两束线偏振光,从空间不同方向出射,因此晶体棱镜是一种偏振镜,其可以被用于起偏或者检偏,其性能优于人造偏振片和玻片堆,因为晶体双折射现象中o光和e光都是完全的线偏振光。下面介绍一些经典的晶体棱镜。

\begin{figure}[htbp]
    \centering
    % 第一张图
    \begin{minipage}[b]{0.48\textwidth}
        \centering
        \includegraphics[width=\textwidth]{4-18.png} % figure/image1.png
        \caption{尼科耳棱镜示意图}
    \end{minipage}
    \hfill
    % 第二张图
    \begin{minipage}[b]{0.48\textwidth}
        \centering
        \includegraphics[width=\textwidth]{4-19.png} % figure/image2.png
        \caption{偏振原理示意图}
    \end{minipage}
\end{figure}

(1)尼科耳棱镜。其由两块方解石棱镜黏合而成,其光轴平行于两个端面,常用的粘合剂是加拿大树胶,其折射率大约为$n_B\approx1.55$,介于棱镜两个主折射率$n_e\approx 1.4864$和$n_o \approx 1.6584$之间。在左侧第一块棱镜传播时,自然光几乎呈现正入射,表观上虽然不发生双折射,e光和o光分别到达界面$AB$,对于e光,其为光疏介质到光密介质,不可能发生全反射。全反射是指,在光由光密介质向光疏介质入射时,由于折射角的正弦值不能超过1,从而导致当入射角达到一定大小后,继续增大入射角,折射光消失,所有的光都经由反射传播的现象。而对于o光来说,是从光密介质到光疏介质,将发生全反射,只要通过合适的调控使得入射角大于临界角。这样我们就可以从入射光透射的方向获取一束偏振光,其振动方向平行于主平面或者主截面。


(2)罗雄棱镜。 第一块棱镜光轴垂直棱镜入射面,第二块棱镜光轴平行表面。当自然光正入射于第一块棱镜时不发生双折射,而到达界面后在第二块棱镜中将发生双折射。这样我们实现了o光和e光的分离,o光会沿着光束垂直前进。

(3)沃拉斯顿棱镜。 其与罗雄棱镜唯一的差别在于,第一块棱镜的光轴平行于入射表面,因而其与第二块棱镜的光轴方向正交。这样一来,第一块棱镜中的o光,进入第二块棱镜后变为e光,其出现的双折射现象导致的空间分离角显著地大于罗雄棱镜。

\insertfig{4-20.png}{罗雄棱镜与沃拉斯顿棱镜}{0.8}

下面我们介绍波晶片。

波晶片是一种产生和检验圆偏振光或者椭圆偏振光的器件,通常由水晶中切割下来的一个厚度均匀且光轴平行入射表面的薄片构成。设一束光正入射,那么表观上不会出现双折射现象,但是由于横平面上两个特征振动$\mb{E_o}$和$\mb{E_e}$的传播速率不同,在经过波晶片后会出现相位差,其结果我们先前已经得到过:
\[
\delta_{oe} = \varphi_o-\varphi_e =\frac{2\pi}{\lambda}d(n_e-n_o)
\]

\insertfig{4-21.png}{波晶片}{0.6}

两种常用的波晶片概述如下。第一种是四分之一波晶片。通过其产生的附加相位差服从:
\[
\delta_{oe} = \pm(2k+1)\frac{\pi}{2}, \quad k = 0,1,2\dots
\]

对正晶体,符号取正,反之取负。其提供的有效相位差为:
\[
\delta_{oe} = \pm\frac{\pi}{2}
\]

四分之一波晶片的厚度最小值为:
\[
d_min = \frac{\lambda}{4\Delta n}, \quad \Delta n = |n_e-n_o|
\]

第二种是二分之一波晶片。通过其产生的附加相位差服从:
\[
\delta_{oe} = \pm(2k+1)\pi, \quad k = 0,1,2\dots
\]

其提供的有效相位差为:
\[
\delta_{oe} = \pi
\]

波晶片的厚度最小值为:
\[
d_min = \frac{\lambda}{2\Delta n}, \quad \Delta n = |n_e-n_o|
\]

波晶片的重要应用是检验和产生圆偏振光和椭圆偏振光。为此,有必要对两种偏振光的振动作一介绍。

设波的传播方向沿$z$轴,则这两种偏振的一般形式可以被写为:
\[
E_x = E_{x0} \cos(\omega t-kz+\varphi_x)
\]
\[
E_y = E_{y0} \cos(\omega t-kz+\varphi_y)
\]

我们定义$\Delta \varphi = \varphi_y-\varphi_x$,那么当两者振幅相等,且$\Delta \varphi = \frac{\pi}{2}$时,其为圆偏振光。当$\Delta \varphi = 0,\pm \pi$时,其为线偏振光。$\Delta = \pm\frac{\pi}{2}$时,为正椭圆偏振光。其余情况可以参考图4.22。

\insertfig{4-26.png}{相位差带来的不同偏振情况}{0.5}

注意到图中有箭头标注方向,这代表着偏振光的旋转方向,可以被分为左旋偏振光和右旋偏振光,如下图所示.

\begin{figure}[htbp]
    \centering
    % 第一张图
    \begin{minipage}[b]{0.28\textwidth}
        \centering
        \includegraphics[width=0.5\textwidth]{4-22.png} % figure/image1.png
        \caption{左旋圆偏振光}
    \end{minipage}
    \hfill
    % 第二张图
    \begin{minipage}[b]{0.7\textwidth}
        \centering
        \includegraphics[width=0.6\textwidth]{4-23.png} % figure/image2.png
        \caption{侧视图}
    \end{minipage}
\end{figure}


\begin{figure}[htbp]
    \centering
    % 第一张图
    \begin{minipage}[b]{0.28\textwidth}
        \centering
        \includegraphics[width=0.5\textwidth]{4-24.png} % figure/image1.png
        \caption{右旋圆偏振光}
    \end{minipage}
    \hfill
    % 第二张图
    \begin{minipage}[b]{0.7\textwidth}
        \centering
        \includegraphics[width=0.6\textwidth]{4-25.png} % figure/image2.png
        \caption{侧视图}
    \end{minipage}
\end{figure}
圆偏振光可以被看作两束振动方向相互垂直,振幅相等相位差为$\frac{\pi}{2}$的线偏振光构成. 左旋圆偏振光可以被表示为:
\[
E_y = E_0\cos(\omega t-kz), \quad E_x = E_0\cos(\omega t-kz+\frac{\pi}{2})
\]
右旋圆偏振光可以被表示为
\[
E_y = E_0\cos(\omega t-kz), \quad E_x = E_0\cos(\omega t-kz-\frac{\pi}{2})
\]

现在我们介绍如何通过四分之一波片检验圆偏振光与椭圆偏振光。

为此,我们首先普遍地考察一束偏振光通过波晶片后,出射光的偏振态。对于这类问题,我们以波晶片的光轴方向为基准,在光束传播的横平面上取定座标架$xy$.$x$轴表示e振动而$y$轴表示o振动。在此我们取定光轴方向平行e光的振动方向。接着以此座标架为参考,确定光束在晶片入射点的相位差$\delta_{oe}$,再根据波晶片的种类确定附加相位差$\delta'_{oe}$.最后考察添加相位差之后的出射光的偏振态。下面我们给出偏振光通过四分之一波片和二分之一波片的一些结果,见图4.27.值得注意的是,当入射光的线偏振方位平行或者垂直波晶片光轴时,出射光仍为线偏振光且偏振方向不变,因为两个正交振动之一的振幅为0,不产生附加相位差。

可以观察到,当线偏振光与光轴的夹角为$\frac{\pi}{4}$时,将变为圆偏振光; 夹角不为0或者$\pi$时,将变为正椭圆偏振光。正椭圆偏振光和圆偏振光将变为线偏振光,斜椭圆偏振光仍然是斜椭圆偏振光。

\insertfig{4-27.png}{偏振片通过波晶片后的偏振态}{0.8}

在自然光条件下,如果我们想要获得圆偏振光,需要一个偏振器和和一个四分之一波片联合作用,并且需要保证偏振片的透振方向与光轴方向$e$之夹角为$\frac{\pi}{4}$.

\begin{noteenv}
    实际上的偏振片并不会标明其透振方向,实际上的波晶片也不会标明其光轴方向。因此我们可以采取下面的实验来确保$\frac{\pi}{4}$夹角的实现。
    \begin{itemize}
        \item 取两个偏振片$P_1,P_2$,安排在光路中,转动其中一个直到消光。此时两偏振片的透振方向正交。
        \item 取一个四分之一波片插入两偏振片之间,此时一般不再消光。转动四分之一波片一周,将出现四次消光状态,消光时说明四分之一波片的光轴平行或者垂直e光振动方向。转动四分之一波片直到消光。
        \item 顺或逆时针转动$P_1$经过$\frac{\pi}{4}$角度,则可以保证透振方向$P_1$与光轴$e$的夹角为$\frac{\pi}{4}$。
    \end{itemize}
\end{noteenv}

使用同样的装置,可以区分圆偏振光和自然光。两者的共同特点是偏振度均为0,且在横平面上的偏振结构具有轴对称性。取一张偏振片面对两种偏振光转动时,出射光强始终不变。为此我们在偏振片前插入一个四分之一波片。

对于圆偏振光,经过四分之一波片后成为一个线偏振光。因此转动后面的偏振片,会出现消光状态。对于自然光,其经过四分之一波片后会变为包含大量长短轴比值不同,彼此不相关的椭圆偏振光的集合,宏观上看仍有轴对称性,因此转动偏振片时不发生消光。这样我们就区分了圆偏振光与自然光。

产生椭圆偏振光时,只需要将我们在产生圆偏振光时所使用的装置,把第一个偏振片$P_1$的透振方向调整为非0,$\pi$,$\frac{\pi}{4}$的任意角度,就可以获得椭圆偏振光。

椭圆偏振光和部分偏振光的共同特点是,其偏振度介于$(0,1)$之间,取一张偏振片面对这两者之一转动时,表现出的变化特点都是有光强的极大和极小,但无消光。为此我们仍然在偏振片前插入一张四分之一波片。

对于椭圆偏振光,当我们调整波片的光轴方向,使得入射的椭圆偏振光的长短轴方向与波晶片的坐标轴$(o,e)$方向一致时,入射的椭圆偏振光是一个正椭圆,经过波片后便成为一个线偏振光,转动偏振片,将存在消光现象。对于部分偏振光,其经过四分之一波片后出射的是大量椭圆偏振光的集合,但其宏观特点是相同的,因此当后面的偏振片转动过程中,有光强变化,但是没有消光。这样我们就区分了椭圆偏振光与部分偏振光。

\begin{noteenv}
    其实验操作可以概述如下:
    \begin{itemize}
        \item 取一偏振片,安排在光路中,转动偏振片,记录光强极大和极小值的角度,此即为椭圆偏振光的长轴和短轴的角度。
        \item 根据获取圆偏振光中的方法,找到四次消光状态对应的两个方向,此即为光轴的两个可能方向,其应该是彼此垂直的。
        \item 将任意一个可能方向与前述的长轴方向对齐。
    \end{itemize}
\end{noteenv}

总结一下,我们检验光的偏振态应该遵循下列步骤。

(1)使用一个偏振片,旋转一周检验有无消光,若有,则为线偏振光。

(2)若旋转一周中光强无变化,则为圆偏振光或自然光; 若光强有极大极小但无消光,则为部分偏振光或者椭圆偏振光。此后插入四分之一波片,按照前述流程区分自然光和圆偏振光或者部分偏振光与椭圆偏振光。

\insertfig{4-28.png}{检验偏振态的流程}{0.5}

\section{偏振光干涉}
\insertfig{4-29.png}{偏振光干涉装置}{0.6}

一个典型的偏振光干涉装置如4.29所示,我们以自然光入射。$P_1$和$P_2$为两个偏振片,$C$为一光学装置,一般为波晶片。

我们先介绍一下可能的一些现象。

(1)若波晶片厚度均匀,单色光入射,那么在输出屏上光强$I_2$均匀分布,若转动$P_2$或者其他元件,则输出光强随之变化。

(2)若波晶片厚度不均匀且单色入射,那么输出光强呈现不均匀分布,若转动$P_2$或者其他元件,强度图样随之发生变化,每当$P_2$转过$\frac{\pi}{2}$角度,亮纹与暗纹就交替一次。

(3)若白光入射,在以上两种情形下,输出场呈现彩色条纹或图样,若转动$P_2$或者其他元件,输出场的色彩分布随之发生变化,每当$P_2$转过$\frac{\pi}{2}$角度,两种互补色便交替变化一次。均匀晶片条件下视场中仍然是均匀的光场。

\insertfig{4-30.png}{求解输出光强}{0.7}

我们对上述现象作一分析。经过$P_1$后自然光变为线偏振光,其振幅矢量为$\bm{A_1}$.进入波晶片后,齐备分解为两个正交的特征振动,其振幅矢量分别为$\bm{A_e}$和$\bm{A_o}$。通过波晶片后,两个正交的特征振动产生了相位差变化,记为$\delta'_{oe}$.而通过$P_2$后光矢量只能是平行于$P_2$透振方向的分量,因此需要进一步分解而获得两个在同一方向上的振幅矢量$A_{2e}$和$A_{2o}$。这两个振动满足同频,稳定相位差且振动方向相同这三个相干条件,且空间上重叠,因而会发生干涉。

因此输出光强会被表示为双光束干涉强度的形式:
\[
I_2(x,y)=A_{2e}^2+A_{2o}^2+2A_{2e}A_{2o}\cdot \cos \delta
\]

振幅分量为:
\[
A_{2e}=A_1\cos\alpha\cos\beta, \quad A_{2o} = A_1\sin\alpha\sin\beta, \quad  A_1^2 = \frac{1}{2}I_0
\]
其中相位差$\delta$由三项组成:
\[
\delta = \delta_{oe}(A)+\delta'_{oe}+\delta'' 
\]
$\delta_{oe}(A)$表示入射波晶片前o光和e光的相位差; $\delta'_{oe}$表示波晶片带来的相位差,$\delta''$表示正交坐标轴向$P_2$投影带来的相位差,它只有两种可能取值:
\[
\begin{cases}
0, & \text{o, e 投影于 } P_2 \text{ 的方向一致} \\
\pi, & \text{o, e 投影于 } P_2 \text{ 的方向相反}
\end{cases}
\]

考察一些特例:当$P_1\perp P_2$时,$\alpha+\beta = \frac{\pi}{2}$. 则此时输出光强为:
\[
I_{out} = I_0^2\cos^2\alpha\sin^2\alpha(1+\cos\delta)
\]
且此时$\delta''=\pi$.当$P_1\parallel P_2$时,$\delta''=0$,如图4.31,图4.32所示.

\begin{figure}[htbp]
    \centering
    % 第一张图
    \begin{minipage}[b]{0.46\textwidth}
        \centering
        \includegraphics[width=0.75\textwidth]{4-31.png} % figure/image1.png
        \caption{$P_1 \perp P_2$}
    \end{minipage}
    \hfill
    % 第二张图
    \begin{minipage}[b]{0.46\textwidth}
        \centering
        \includegraphics[width=0.625\textwidth]{4-32.png} % figure/image2.png
        \caption{$P_1 \parallel P_2$}
    \end{minipage}
\end{figure}
因此,我们重新来考察一下这个式子:
\[
I_2(x,y)=A_{2e}^2+A_{2o}^2+2A_{2e}A_{2o}\cdot \cos \delta
\]
我们假定起始状态下:$P_1\perp P_2$,且$\delta = 2k\pi$,$\alpha=\beta=\frac{\pi}{4}$. 则上述式子可以被化简为:
\[
I_2=(A_{2e}+A_{2o})^2=A_1^2\cos^(\alpha-\beta)
\]
因此,当我们把$P_2$向光轴方向旋转时,在$P_2$未越过光轴前,$\delta = 2k\pi$保持不变,那么光强会逐渐减小。在$P_2$越过光轴后,$\delta$会产生$\pi$的相位突变,于是从而使得光强变为:
\[
I_2=A_1^2\cos^2(\alpha+\beta)
\]
在$P_1\parallel P_2$时,光强将减弱到0.

以上我们讨论的都是单色光的入射。考虑复色光入射,注意到波片的所谓四分之一等必须指定光的波长,因为:
\[
\delta_{oe} = \frac{2\pi}{\lambda}(n_e-n_o)
\]
中包含了波长,且根据色散理论可能o光和e光的折射率也不尽相同。因此根据我们上面的讨论,可能存在一个角度,使得对于$\lambda_1$波长的光干涉后光强极大,对$\lambda_2$的光干涉后光强极小,这样干涉屏上将出现不同颜色的干涉场。这被称为色干涉。

作为本章的结尾,我们来讨论杨氏双缝干涉中的偏振光现象。

\insertfig{4-33.png}{杨氏双缝干涉与偏振光}{0.6}

如图所示,在准单色面光源照明下,单孔$Q$成为一准单色点光源,发出自然光,照明双孔 $(\text{Q}_1, \text{Q}_2)$,使其成为一对相干点源,在屏幕上产生一组干涉条纹. 现在,分别在不同位置插入偏振片 $\text{P}_0$ 或 $(\text{P}_1, \text{P}_2)$ 或 $\text{P}'$,我们讨论下列条件下屏幕上干涉场的变化.

\begin{enumerate}
    \item [(1)] 仅有偏振片 $\text{P}_0$.
    \item [(2)] 仅有 $(\text{P}_1, \text{P}_2)$,且 $\text{P}_1 // \text{P}_2$.
    \item [(3)] 仅有 $(\text{P}_1, \text{P}_2)$,且 $\text{P}_1 \perp \text{P}_2$.
    \item [(4)] 有 $(\text{P}_1, \text{P}_2)$ 和 $\text{P}'$,且 $\text{P}_1 \perp \text{P}_2$.
    \item [(5)] 有 $\text{P}_0, (\text{P}_1, \text{P}_2)$ 和 $\text{P}'$,且 $\text{P}_1 \perp \text{P}_2$.
\end{enumerate}

我们可以把自然光分解为两个彼此正交的振动$E_x(t)$ 和 $E_y(t)$ ,它们之间是完全非相关的,即两者的相位差是个随机量,同理,由点源 Q 控制的两个点源 $(\text{Q}_1, \text{Q}_2)$ 也是那样,即 $E_{1x}(t)$ 与 $E_{1y}(t)$ 之间、$E_{2x}(t)$ 与 $E_{2y}(t)$ 之间的相位差是随机量,但是,$E_{1x}(t)$ 与 $E_{2x}(t)$ 之间、$E_{1y}(t)$ 与 $E_{2y}(t)$ 之间的相位差是稳定的. 据此,我们对原先无任何偏振片时,杨氏实验干涉条纹的生成机制可认为是:最终出现于屏幕上的是这两套分布完全相同的干涉条纹的非相干叠加,亮纹亮度相较于仅有一组时增加了一倍.

若只有偏振片 $\text{P}_0$,这相当于只保留了上述一套干涉条纹,亮纹亮度减弱为一半. 或者说,因为有了 $\text{P}_0$,只提取了自然光总光强的一半进入干涉系统,致使后场的所有光强效应也减弱一半.

若仅有 $(\text{P}_1, \text{P}_2)$,且 $\text{P}_1 // \text{P}_2$,亮纹亮度减弱一半;若 $\text{P}_1 \perp \text{P}_2$,单凭正交振动必定是非相干叠加这一条,就可以断定屏幕上亮度均匀,干涉条纹消失。

那么,在 $\text{P}_1 \perp \text{P}_2$ 条件下,若在屏幕前面加一偏振片 $\text{P}'$,以提取两个振动方向一致的成分,是否就可以发生相干叠加而生成干涉条纹呢?实验结果显示,屏幕上亮度依然均匀而无干涉条纹. 这时,同频同振动方向这两条相干条件,虽然得以保证,但稳定相位差这一条件未能实现,因为从 $\text{P}_1, \text{P}_2$ 透射过来的是自然光的两个正交振动,比如 $E_{1x}(t), E_{2y}(t)$,两者之间的相位差是一随机量,即使各自向 $\text{P}'$ 投影获得同方向的两个分量,其间相位差显然还是随机量,不满足相干条件. 这一现象就是历史上著名的阿拉戈偏振光干涉实验所显示的,他于 1816 年发现,从自然光中提取的偏振方向互相垂直的两束光不干涉,即使外加一偏振片在其交叠区中.

在 $\text{P}_1 \perp \text{P}_2$ 条件下,若要出现干涉条纹,必须在前面插入一偏振片 $\text{P}_0$ 于 Q 处,在后面再插入一偏振片 $\text{P}'$ 于屏幕前. 前者为了保证有稳定的相位差,因为线偏振光的两个正交分量之间相位差是固定的. 

\begin{thebibliography}{99}

% 1. 现代光学基础
% 版本说明:第2版出版于2012年,作者钟锡华,北京大学出版社
\bibitem{zhong2012}
钟锡华. 现代光学基础[M]. 2版. 北京: 北京大学出版社, 2012.

% 2. 新概念物理教程:光学
% 版本说明:第2版出版于2021年(最新版),作者赵凯华,高等教育出版社
\bibitem{zhao2021}
赵凯华. 新概念物理教程: 光学[M]. 2版. 北京: 高等教育出版社, 2021.

% 3. 大学物理教程 AR版
% 版本说明:根据胡其图教授在人民邮电出版社发行的版本,上册出版于2023年,下册出版于2024年
% 此处引用整套教材,年份标记为起始年份 2023
\bibitem{hu2023}
胡其图, 李晟, 刘世勇. 大学物理教程: AR版[M]. 北京: 人民邮电出版社, 2023.

% 4. 光学(重排本)
% 版本说明:2018年重排版,作者赵凯华、钟锡华,北京大学出版社
\bibitem{zhao2018}
赵凯华, 钟锡华. 光学: 重排本[M]. 北京: 北京大学出版社, 2018.

\end{thebibliography}


\end{document}
